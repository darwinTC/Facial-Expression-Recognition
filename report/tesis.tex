
\documentclass[12pt]{report}
\usepackage[utf8]{inputenc}
%\usepackage[utf8x]{inputenc} 
%\usepackage[brazilian]{babel}
%library for algorithms
\usepackage{algorithm}
\usepackage[noend]{algpseudocode}


\usepackage[spanish]{babel}
\usepackage{graphicx}
\usepackage{hyperref}
\usepackage{multirow} % para las tablas
\usepackage{amsmath}
\usepackage{amssymb}
\usepackage{url}
\usepackage{array}
\usepackage{tabularx}
\usepackage{setspace}
\usepackage{geometry}
\usepackage{verbatim}
\usepackage{caption}
\geometry{top=2.5cm,bottom=2.2cm,left=2.5cm,right=2.5cm}
\usepackage{endnotes}
\usepackage{hyperref}
\usepackage{float}
%\usepackage{glossaries}
\usepackage[acronym]{glossaries}
\usepackage[nocompress]{cite}
\usepackage{float}
\usepackage{multicol}
\makeglossaries

%--------------
\makeatletter
\def\BState{\State\hskip-\ALG@thistlm}
\makeatother




%--------- for confusion matrix
\newcommand\MyBox[2]{
  \fbox{\lower 1cm
    \vbox to 2cm{\vfil
      \hbox to 2cm{\hfil\parbox{1.6cm}{#1\\#2}\hfil}
      \vfil}%
  }%
}

%------------






%--------------
\newcommand{\Px}{\mathbf{x}}
\newcommand{\PX}{\mathbf{X}}
\renewcommand{\sin}{\qopname\relax o{sen}}

\newenvironment{myenumerate}{
	\begin{enumerate}
		\setlength{\itemsep}{5pt}
		\setlength{\parskip}{0pt}
		\setlength{\parsep}{5pt}
	}{\end{enumerate}}

\newenvironment{myitemize}{
	\begin{itemize}
		\setlength{\itemsep}{5pt}
		\setlength{\parskip}{0pt}
		\setlength{\parsep}{5pt}
	}{\end{itemize}}

%--------------------------------------------



\begin{document}
\tolerance=999
\sloppy
\renewcommand\bibname{BIBLIOGRAFÍA}

%--------------------------------------caratula
\pagenumbering{Roman} % para comenzar la numeracion de paginas en numeros romanos
\thispagestyle{empty} %oculta la numeracion

\begin{center} {\large \bf UNIVERSIDAD NACIONAL DE SAN ANTONIO ABAD DEL CUSCO} \end{center}
\begin{center} {\large \bf FACULTAD DE INGENIERÍA ELÉCTRICA, ELECTRÓNICA, INFORMÁTICA Y MECÁNICA} \end{center}
\begin{center} {\large  ESCUELA PROFESIONAL DE INGENIERÍA INFORMÁTICA Y DE SISTEMAS} \end{center}
\vspace{0.5cm}
\begin{figure}[htb]
\centering
\includegraphics[width=33mm]{Imagenes/simbolo_unsaac.png}
\end{figure}
\vspace{0.5cm}
\begin{center} {\large \textbf{TESIS DE INVESTIGACIÓN}} \end{center}
\vspace{0.5cm}
\begin{center}  {\textsc{\textbf{DESARROLLO DE UNA ARQUITECTURA DE RED NEURONAL
CONVOLUCIONAL COMO UN MODELO DEL PROCESO
CEREBRAL HUMANO PARA LA CLASIFICACIÓN DE
EXPRESIONES FACIALES}}}

\vspace{1.5cm}

{\textbf{Para optar al título profesional de}: \\Ingeniero Informático y de Sistemas\\[0.2cm ]
}

{\textbf{Presentado por}: \\
\hspace{-0.25cm} Br. Darwin Ttito Concha \\ 
\hspace{0.5cm} Br. Paul Dany Flores Atauchi \\[0.2cm ]
\textbf{Asesor}: \\Prof. Msc. Lauro Enciso Rodas \\[0.2cm]
\textbf{Co-Asesor:} \\Prof. M.Eng E. Gladys Cutipa Arapa \\[0.2cm]
}

\vspace{1.0cm}
\begin{center} {\tiny  \textbf{FINANCIADO POR EL CONSEJO DE INVESTIGACIÓN DE LA UNSAAC}} \end{center}
\begin{center} {CUSCO - PERÚ\\
2017} \end{center}
\end{center}
%------------------------------------------------------------------

\clearpage

\begin{spacing}{1.05} %interlineado

\chapter*{Dedicatoria}

\textit{Dedico esta tesis a mi mamá Luz Marina, a mi tía Felicitas Magdalena y a mis hermanos Sharmely y Russel por el gran apoyo y motivación que siempre me brindan} 
\begin{flushright}\textit{Paul Dany Flores Atauchi}\end{flushright}

\vspace{1cm}
\textit{Este trabajo está dedicado a mis padres Marina y Donato, mis hermanos Edison y Sayda. A todos ellos porque día a día me aconsejan y ayudan a ser una mejor persona.}
\begin{flushright}\textit{Darwin Ttito Concha}\end{flushright}

%\chapter*{Resumen}
%Las expresiones faciales son un medio de comunicación no verbal por el que los seres humanos transmiten sus emociones en el proceso de interacción con su entorno. Esta interaccion  puede darse con otros seres humanos o puede ser una reacción frente a estimulos externos de su entorno como la publicidad o reacción frente a un servicio consumido o por consumir. Existen muchas aplicaciones útiles en el mundo real que son derivadas del reconocimiento automático de expresiones faciales tales como: Estudio de marketing, interacción hombre-máquina, análisis psicologico, seguridad, etc. Muchos abordages tradicionales aplicados al reconocimiento de expresiones faciales tienen dificultad en la extracción y representación de características de la imagen. Esta dificultad se debe a que estos metodos diseñan la extracción de caracteristicas de forma manual y la representación de características no toma en consideración el relacionamiento global dentro de toda la imagen. Tanto la extracción como representación de características de una imagen son parte importante para construir un buen clasificador. Recientemente, las técnicas de deep learning estan logrando resolver las dificultades de los métodos tradicionales en muchas tareas de visión por computador. Especificamente, las redes neuronales convolucionales logran superar el problema de extracción y representación de características de una imagen. En este trabajo, proponemos una arquitectura de red neuronal convolucional para la clasificación de expresiones faciales. Consideramos seis categorías para clasificar las expresiones faciales: Enojo, miedo, alegría, tristeza, sorpresa y neutro. Este trabajo aporta a una mejor comprensión sobre las redes neuronales Convolucionales aplicada al reconocimiento de expresiones faciales e imágenes en general, también ayudara en el desarrollo de futuros proyecto que necesiten del reconocimiento de expresiones faciales.
%\begin{center}
%\noindent\rule{16cm}{0.5pt}
%\end{center}
%\textbf{Palabras Claves:} Expresiones faciales, redes neuronales convolucionales , reconocimiento de Patrones.



\chapter*{Resumen}
Las expresiones faciales son un medio de comunicación no verbal que muestran las emociones de una persona, estas expresiones ayudan a transmitir información en las interacciones inter personales y facilitan el entendimiento del significado del lenguaje hablado. Por lo que se considera que poder clasificar la expresión de un rostro sería una gran fuente de información para una posterior utilización. El objetivo del presente proyecto es modelar el proceso cerebral humano para clasificar imágenes de expresiones faciales por medio de una de las técnicas de \textit{Deep Learning}, logrando así que una máquina sea capaz de aprender de imágenes de expresiones faciales suministradas de ejemplo (datos de entrenamiento) con el objetivo de poder clasificar ejemplos futuros sin ningún tipo de intervención humana en el proceso. En la actualidad, gracias a las Redes Neuronales Convolucionales, se están logrando buenos resultados en la clasificación de imágenes, detección de objetos, comprensión de escena, en comparación con técnicas convencionales, por lo cual en este proyecto se usó la arquitectura de una Red Neuronal Convolucional para clasificar las expresiones faciales, clasificándolas en 6 categorías: enojo, miedo, alegría, tristeza, sorpresa y neutro. Este trabajo aporta a una mejor comprensión en las redes neuronales Convolucionales aplicada al reconocimiento de expresiones faciales e imágenes en general, también ayudara en el desarrollo de futuros proyecto que necesiten del reconocimiento de expresiones faciales, como: estudio de marketing, interacción hombre-máquina, psicología, análisis educativo y otros.
\begin{center}
\noindent\rule{16cm}{0.5pt}
\end{center}
\textbf{Palabras Claves:} Expresión Facial, \textit{Deep Learning}, Convolutional Neural Networks, Visión por Computador, Reconocimiento de Patrones, Detección de Objetos.
\chapter*{Abstract}
Facial expressions are a means of nonverbal communication that show emotions of people, these expressions help interpersonal transmit information and facilitate the understanding of the meaning of spoken language. So that we believe that to determine the facial expression would be a rich source of information for later use. The objective of this project is to simulate the behavior our brains to recognize images of facial expressions using one of the techniques of Deep Learning, achieving a machine can learn from images of facial expressions supplied sample (training data) in order to classify future examples, without any human intervention in the process. Nowadays thanks to the technique used in this project (convolutional neural network) researchers are achieving good results in image recognition, object detection, understanding scene, compared with other conventional techniques, so in this project use the basic architecture of a convolutional neural network to recognize facial expressions, and classified into 6 categories: happiness, sadness, joy, fear, anger and surprise. This paper give us more understanding on convolutional neural network applied to the recognition of facial expressions and images in general also help in the development of future project requiring the recognition of facial expressions like systems man-machine interaction, marketing analysis based on the facial expressions of people, behavioral studies, mental health and cognitive processes.
\begin{center}
\noindent\rule{16cm}{0.5pt}
\end{center}
\textbf{Key Words:} Facial Expression Recognition, Understanding Scene, Object Detection, Convolutional Neural Network, Deep Learning.


%----------------------------------------indices
\renewcommand{\contentsname}{Índice General}
\renewcommand{\listfigurename}{Índice de Figuras}
\renewcommand{\listtablename}{Índice de Tablas}
\renewcommand\tablename{Tabla}
\tableofcontents

\cleardoublepage
\addcontentsline{toc}{chapter}{Indice de Figuras} % para que aparezca en el indice de contenidos
\listoffigures

\cleardoublepage
\addcontentsline{toc}{chapter}{Indice de Tablas} % para que aparezca en el indice de contenidos
\listoftables


%--------------------------------------------


\clearpage
\pagenumbering{arabic}
\setcounter{page}{1}

\chapter*{Introducción}
Uno de los razgos psicológicos naturales de los seres humanos es la necesidad de interacción entre ellos. La interacción puede darse a través de la comunicación. Propio de éste, es generada y obtenida información que ellos usan para tomar decisiones durante el proceso de comunicación. Actualmente, gracias al gran avance tecnológico vivimos en una era digital que, posibilitó la trascendencia de la comunicación directa a la comunicación indirecta masiva a través de las redes sociales, videoconferencias, etc. Producto de ello, la cantidad de datos (imágenes, videos, textos, etc) generada por minuto en internet es exorbitante. Surgiendo muchas potenciales aplicaciones (reconocimiento de expresiones faciales, reconocimiento de objetos, detección de objetos, etc) que implican el análisis de imágenes y/o videos por computador. 

Las expresiones faciales juega un rol importante en la comunicación no verbal entre los seres humanos, además de ser el medio por el que se transmite más del 55\% de información en el proceso  de comunicación. Para el reconocimiento de las expresión facial por medio del computador son diseñados modelos basados en la apariencia y modelos geométricos del rostro humano.
Así, en este trabajo abordamos el reconocimiento de expresiones faciales como un problema de clasificación de imagenes de rostros humanos clasificados en 6 categorias (alegre, neutro, feliz, triste, enojado, sorpresa) basados en la apariencia del rostro.

La metodología propuesta  para abordar el problema de reconocimiento de expresiones faciales en imágenes es compuesta por 3 partes. La primera parte, comprende la estandarización y la normalización de las imagenes. En la estandarización, las imagenes son redimensionadas a un tamaño estandar y convertidas a tonos de cinza. La normalización es aplicada para reducir la varianza en las imágenes. Para ambos casos se usa métodos y técnicas de procesamiento de imagenes. En la segunda parte, se extrae las caracteristicas del rostro y se clasifica en las diferentes clases de expresiones faciales (alegre, neutro, feliz, triste, enojado, sorpresa). Para ello, se diseña un conjunto de arquitecturas de redes neuronales convolucionales (CNNs) con diferentes parámetros. Finalmente, se evalua el modelo con mayor rendimiento en 3 bases de datos públicos de expresiones faciales (FER2013, CK+) y un adicional creado a partir de la unión de los dos antes mencionados (Fer2013-CK+). 

Adicionalmente, para comprobar el funcionamiento del modelo en imágenes del mundo real, un pequeño conjunto de imágenes de internet es seleccionado por nosotros. Estas imágenes possen contenidos variados desde imágenes con ruido, rostros con oclusión, fondo complejo hasta imágenes de animes. Primero, es extraido el rostro humano via el detector de rostros \textit{haar cascade} y posteriormente es reconocido su correspondiente expresion facial usando el modelo creado anteriormente.

El desarrollo del presente trabajo puede resumirse en 3 partes.

\begin{itemize}
\item \textbf{Parte I:} Cubre los aspectos generales del problema, describiendo de una manera detallada el problema al cual se quiere dar solución, los trabajos relacionados, los objetivos a alcanzar, la metodología y las limitaciones encontradas en el desarrollo de la investigación.
\item \textbf{Parte II:} Proporciona los fundamentos teóricos necesarios que son vitales para el desarrollo y entendimiento del proyecto.
\item \textbf{Parte III:} Desarrolla y muestra los experimentos realizados con diferentes configuraciones sobre una arquitectura de red neuronal convolucional(CNN). Se describe a detalles el funcionamiento de los métodos elegidos para la deteccién y el reconocimiento de expresiones faciales. Los resultados obtenidos son interpretados en terminos de una métrica de error y precisión, y se proponen trabajos futuros.

 %Parte III
 %Describe el desarrollo del metodo propuesto, asi como los experimentos realizados en las diferentes configuraciones de las arquitecturas de Redes Neurales convolucionales (CNNs) propuestas. Se describe a detalle el funcionamiento del metodo de clasificacion de expresiones faciales. Los resultados obtenidos son interpretados basados en el error y precision del modelo en las diferentes bases de datos de expresiones faciales. Por ultimo las conclusiones, recomendaciones y trabajos futuros son mostrados.
 
 % Para la aplicacion adicional descrita anteriormente, se describe detalladamente el funcionamiento del metodo de deteccion de rostros. 
 
  
\end{itemize}







\part{Aspectos Generales}
\chapter{Aspectos Generales}
\section{Aspectos Generales}
\subsection{Descripción del Problema}
Uno de los sentidos más importantes de los seres humanos es la visión. Ésta es empleada para obtener la información visual del entorno físico. Según Aristóteles, “Visión es saber que hay y donde mediante la vista”. De hecho, se calcula que más del 75\% de las tareas del cerebro son empleadas en el análisis de la información visual. El refrán popular de “Una imagen vale más que mil palabras” tiene mucho que ver con los aspectos cognitivos de la especie humana. Casi todas las disciplinas científicas emplean utillajes gráficos para transmitir conocimiento\footnote[1]{Visión Humana, fuente: Sistemas Adaptativos y Bioinspirados en Inteligencia Artificial\href{http://sabia.tic.udc.es/}{(S.A.B.I.A.)}}.

Uno de los más grandes concursos a nivel mundial en clasificación de imágenes reporta que las técnicas tradicionales (como las técnicas de extracción de características en imágenes estáticas: \textit{Principal component analysis} (PCA), \textit{Edges detector}, \textit{Gabor waveled}; Video: PCA, \textit{Discrete cosine transform} (DCT), \textit{optical flow} e \textit{image difference}) están siendo superadas por técnicas de \textit{Deep Learning} que son basadas en el proceso cerebral humano. Dicho éxito se debe a que las técnicas tradicionales requieren de un ambiente controlado y no son tolerables a cambios como: traslación, rotación y escalado. Por otro lado las técnicas de \textit{Deep Learning} demuestran ser mas robustas y efectivas frente a estos tipos de cambios\footnote[2]{\textit{Deep Learning} vs. \textit{Machine Learning} fuente: \href{http://www.image-net.org/}{Analytics Vidhya}}.
	
Diversas actividades cotidianas necesitan del reconocimiento de imágenes, tal es el caso del reconocimiento de expresiones faciales, que en los últimos años se ha convertido en una de las tareas más estudiadas por investigadores en todo el mundo, con el fin de alcanzar un margen de error minimo para posteriormente centrarse en el desarrollo de aplicaciones en distintos campos como: estudio de marketing, interacción hombre-computador, psicología y análisis educativo \footnote[3]{Las expresiones faciales de las emociones, historia y aplicaciones, fuente: \href{http://medina-psicologia.ugr.es/cienciacognitiva/?p=664}{Ciencia Cognitiva}}, las cuales han sido abordada por diferentes técnicas tradicionales no obteniendo resultados prometedores en imágenes reales que contienen distintos tipos de variaciones (mencionados en el párrafo anterior). Limitando así, la implementación y desarrollo de aplicaciones útiles para el bien común (aplicaciones antes mencionadas).

\subsection{Identificación del Problema}
Las técnicas tradicionales para el reconocimiento de expresiones faciales usadas en la actualidad necesitan de un ambiente controlado (iluminación constante, alta calidad de imagen, poco ruido, imagen sin oclusión), y no son tolerables a cambios como rotación, traslación, escalado. Limitando así, la creación de aplicaciones con imágenes del mundo real (imágenes obtenidas a partir de cámaras de seguridad). Por lo que hay la necesidad de usar nuevas técnicas del estado del arte que nos permitan obtener mejores resultados superando así la limitación antes mencionada.

\section{Antecedentes}
Se muestra una lista de trabajos resaltantes que hacen uso de técnicas de \textit{deep learning}, los cuales sirvieron de inspiración y fuente de información valiosa en el desarrollo de este trabajo. También se presentan trabajos dedicados al reconocimiento de expresiones faciales utilizando técnicas tradicionales de visión por computador y \textit{machine learning}.

\subsection{Técnicas Tradicionales para el reconocimiento de expresiones faciales}
Muchos abordages fueron propuestos para la tarea de reconocimiento de expresiones faciales basados técnicas tradicionales y \textit{machine learning}. 

\begin{enumerate}
\item {\textbf{“Learning active facial patches for expression analysis” \cite{zhong2012learning}}

Zhong et al. propone un método para el reconocimiento de expresiones faciales basado en \textit{patches} informativos de expresiones faciales (e.g. ojos, mejillas, boca, etc) de los rostros humanos contenidos en las imágenes. Así, introduce un marco de aprendizaje por tareas multiples (MTSL) de dos etapas para ubicar de manera eficiente los \textit{patches} discriminativos en las imagenes con contenido facial. La primera etapa consiste en extraer un conjunto de \textit{patches} por cada expresión facial dentro de un conjunto de datos de entrenamiento. En la segunda parte del MTSL, extraído el conjunto  de \textit{patches} por cada expresion facial en el conjunto de datos de entrenamiento dos tareas son desarrolladas, el reconocimiento de expresión facial y verificación facial.  Estas dos tareas son integrados para aprender de los \textit{patches} obtenidos con anterioridad. El aprendizaje tiene como objetivo determinar cuales son los específicos \textit{patches} faciales que corresponden a cada expresión facial. Finalmente, un clasificador SVM es entrenado para reconocer las características contenidas en los \textit{patches} para una de las seis expresiones faciales consideradas en este trabajo. 
}

\item{\textbf{“Local gabor binary patterns from three orthogonal planes for automatic facial expression recognition” \cite{almaev2013local}}

Almaev et al.  propone un descriptor dinámico de apariencias \textit{Local Gabor Binary Patterns de Three Ortogonal Planes}(LGBP-TOP) donde, analizando la textura espacial y dinámica, y combinado con filtros de gabor logra un alto nivel de precisión en el reconocimiento de expresiones faciales en tiempo real. Además, el descriptor propuesto es robusto a errores en la alineación rotacional propios de la captura de registros.
}

\item{\textbf{“An integrated approach for efficient analysis of facial expressions” \cite{ghayoumi2014integrated}}

Ghayoumi et al.  propone la integración de tres métodos para el reconocimiento de expresiones faciales. \textit{Locality Sensitive Hashing} (LSH), \textit{Principal Component Analysis} (PCA) y \textit{Linear Discriminant Analysis} (LDA)  son los tres métodos utilizados para el reconocimiento de expresiones faciales en este trabajo. Para reconocer una expresión facial son extraídos los vectores de características de dos regiones más informativas del rostro, la región de los ojos y la región de la boca. El LSH esta compuesto por funciones hash los cuales son usados para mapear los vectores de características extraídos de las imagenes en bloques de colisiones, donde cada bloque esta registrado como la locación de vectores de características pertenecientes a una específica expresión facial a priori. Así, se logra reducir redundancia en el espacio de representación de las imagenes respecto a las imágenes que pertencen al mismo tipo de expresion facial. Luego de este paso son aplicados los métodos de reducción de dimensionalidad PCA y LDA  para aliviar la complejidad computacional y reducir redundancia en los datos. Por último, es entrenado un clasificador SVM con los datos obtenidos despues de aplicado los métodos de reducción de dimensionalidad.
}

\end{enumerate}

\subsection{Técnicas de \textit{Deep Learning}}

Muchos de los avances en visión por computador, procesamiento del lenguaje natural, bioinformática y otros campos de investigación se debe a la utilización de métodos y técnicas de inteligencia artificial, específicamente técnicas de \textit{machine learning} y \textit{deep learning}. Este último, es el causante de una gran impacto en el avance técnologico con relacionamiento directo en el mundo comercial (p.e. carros autónomos, chatboots, creación de nuevos farmacos por computador, etc). 

\begin{enumerate}

\item{\textbf{“Gradient-based learning applied to document recognition” \cite{2lecun1998gradient}}

 Yan Lecunn  et al. introduce uno de los primeros trabajos con redes neuronales convolucionales, un tipo especial de \textit{deep learning}. En este trabajo, muestra la potencialidad de la técnica de aprendizaje basado en gradiente. Esta técnica es utilizado en el entrenamiento de una \textit{multilayer perceptron} via el algoritmo de \textit{backpropagation}. Es diseñado una red neuronal convolucional para tratar con la variabilidad de forma 2D en imágenes para el reconocimiento de dígitos escritos a mano. También es introducido dos sistemas on-line para el reconocimiento de documentos escritos a mano. Para controlar los componentes del sistema tales como la extracción, segmentación, reconocimiento y modelage del lenguaje; es introducido un nuevo paradigma de aprendizaje llamado \textit{Graph Transformer networks}. Además, es descrito la pontencialidad del \textit{Graph transformer networks} aplicado a la lectura de cheques de banco. Donde, las redes neuronales convolucionales para el reconocimiento de caracteres combinado con técnicas de entrenamiento global proporcionaron un alto rendimiento. Esto fue demostrado en el area comercial al ser capaz el modelo de leer cheques personales de forma automatizada, logrando leer millones de cheques por día. 
}
\item{\textbf{“Imagenet classification with deep convolutional neural networks” \cite{8krizhevsky2012imagenet}}
 
Krizhevsky et al. propone una arquitectura de red neuronal convolucional profunda que es entrenado para clasificar 1.2 millones de imágenes en alta resolución en 1000 categorías. La red propuesta en este trabajo llega a estar compuesto por 60 millones de parámetros y 650000 neuronas. Para el entrenamiento fueron usados \textit{GPUs} para acelerar el computo de multiplicaciones matriciales requeridas en las operaciones de convolución. Este trabajo logra el estado del arte en clasificación de imágenes a gran escala en la competición ILSVRC-2012. Esto es demostrado al lograr reducir el error de clasificación de imágenes en la base de datos ImageNet de 26.2\% a 15.3 \%. Siendo así, uno de los trabajos con más impacto en visión por computador. 

Así, es demostrado que una de las principales desventajas de los métodos basados en extracción de características diseñadas a mano es el tiempo de computo lo que les dificulta ser escalables, requisito principal para aplicaciones en el mundo real. además, estos métodos solo cubren una parte específica de casos, esto debido a las suposiciones que se toman para casos específicos que no logran cubrir todo el espectro de posibilidades para poder crear modelos generalizables. Por otro lado, los métodos de \textit{deep learning} demostraron ser capaces de crear modelos generalizables y escalables. Hecho este análisis, fuimos motivados a desarrollar una arquitectura de red neuronal convolucional para el reconocimiento de expresiones faciales.
}
\end{enumerate}
\section{Objetivos}
\subsection{Objetivo General}
Desarrollar una arquitectura de red neuronal convolucional que sea capaz de obtener niveles de precisión confiables (mínimo margen de error) en el reconocimiento de expresiones faciales, permitiendo así contribuir en el desarrollo de futuras aplicaciones del mundo real que sirvan para el beneficio de la sociedad.
\subsection{Objetivos Específicos}
\begin{itemize}
\item Selección de las bases de datos de expresiones faciales y el pre-procesamiento de estos datos (estandarización y normalización).

\item Investigar los filtros de convolución para la correcta selección de los parámetros.

\item Investigar la función de submuestreo para la correcta selección de los parametros. 

\item Investigar las funciones de activación y funciones de normalización para la correcta selección de los parametros.

\item Diseñar la arquitectura propuesta (configuración de parametros, número de capas y funciones de activación y normalización), basandonos en los objetivos previos.

\item Entrenar la arquitectura propuesta.
	
\item Evaluar el modelo creado a partir de la arquitectura propuesta.

\item Analizar e interpretar los resultados
\end{itemize}

\section{Alcances}
En este trabajo de investigación se lograron los siguientes alcances.

\begin{itemize}
\item Se propuso una nueva arquitectura para el reconocimiento de expresiones faciales, basada en las redes neuronal convolucional. El modelo creado fue capaz de obtener altos niveles de precisión que serán de utilidad para el desarrollo de futuras aplicaciones en el mundo real.
\item Se creo una nueva base de datos, la cual fue resultado de la unión de las dos bases de datos antes mencionadas(FER2013 y CK+).
\item Contribuimos con la comunidad académica del país y la región brindandoles información de un tema de investigación actual que servirá como base para el desarrollo de futuras aplicaciones y trabajos relacionados.
\end{itemize}
 

\section{Justificación}
En la actualidad se ha dado más realce a algunas disciplinas de la inteligencia artificial como: \textit{machine learning} y \textit{deep learning}, disciplinas que brindan distintas técnicas que están dando solución a problemas de clasificación de imágenes, comprensión de escena, análisis de sentimientos y otros. Así, es el caso de la visión artificial donde las redes neuronales convolucionales está proporcionando mejores resultados en comparación con algoritmos y técnicas tradicionales.

En este trabajo, presentamos un estudio resumido de la investigación hecha en \textit{deep learning} con aplicación en el reconocimiento de expresiones faciales que servirá tanto para los investigadores como para los lectores. También este trabajo ayudara para el desarrollo de futuros proyectos de clasificación de imágenes en distintos campos (seguridad, medicina y biología, internet y la nube, entretenimiento, máquinas autónomas y otros)\footnote[4]{Aplicaciones de Deep Learning, fuente: \href{https://developer.nvidia.com/deep-learning}{NVIDIA} GPUs - el motor del aprendizaje profundo(deep learning)}.


\section{Metodología}
Dada la naturaleza del trabajo de investigación, se utilizó los métodos de investigación bibliográfica, explorativa y aplicativa. Bibliográfica ya que se recogió y analizó información para obtener conocimientos previos sobre \textit{deep learning} y el detector \textit{haar cascade}. Explorativa porque se seleccionó información relevante procedente de la etapa de investigación bibliográfica, que sirvió para la construcción de la arquitectura de una \textit{red neuronal convolucional} basándonos en trabajos previos relacionados con la línea de investigación. Aplicativa por que se utilizaron los conocimientos adquiridos \cite{19sabino1994hacer}\cite{23silva2001metodologia}.


\section{Limitaciones}
\begin{itemize}
\item Dificil acceso a herramientas tecnológicas de hardware, principalmente GPU's de alta capacidad, necesarias para la fase de entrenamiento de la red neuronal convolucional. 

\item Dificil acceso a información cientifica de fuentes confiables (IEEE, ACM, Springer, etc). Esto debido al elevado costo para acceder a ellos.


\item Carencia de organizaciones peruanas que brinden grandes bases de datos de expresiones faciales para poder utilizarlos en la fase de entrenamiento y crear un modelo para el reconocimiento de expresiones faciales de la región o del país. Así, fuimos llevados a utilizar bases de datos de organizaciones extranjeras que fomentan la investigación en esta área.


% ya que se requiere miles de imágenes (imágenes de acuerdo al proyecto de investigación en el que se trabaje, como: rostros, danzas, señas, sitios arqueológicos y otros) para que se cree un modelo eficiente y robusto, llevando a utilizar base de datos de organizaciones extranjeras que fomentan la investigación en esta área.
\end{itemize}
\newpage
\section{Cronograma de Actividades}
\begin{table}[!htb]
    \centering
    \includegraphics[angle=90,width=50mm]{Imagenes/cronograma.png}
    \caption{Cronograma de actividades}
    \label{tab:tab1}
\end{table}


 

\part{Marco Teórico}
\chapter{Marco Teórico}
\section{Conceptos de Visión}

\subsection{Visión Humana}
De una manera muy generar, vision se entiendo como toda accion de ver, sin embargo, desde un punto de vista mas tecnico, vision es la capacidad de interpretar nuestro entorno gracias a los rayos de luz que alcanzan el ojo. Otros autores definen vision como una capacidad necesaria mas no impresindible para realizar las actividades cotidianas.

Desde el punto de la vista de la medicina, la visión humana o sentido de la vista se reduce a un organo receptor conocido como el \textit{ojo}, la membrana y retina son los encargados de recivir las impresiones luminosas para luego transmitirlas al cerebro por medio de las vias opticas(ver figura\ref{fig:estructura_percepcion}). En adicion, el ojo es un organo situado en la cavidad orbitaria, esta protegida por los parpados y por la secrecion de las glandulas lagrumales. 

Los ojos son sensibles a ondas de radiación electromagnética de longitudes específicas. Estas ondas se registran como la sensación de la luz. Cuando la luz penetra en el ojo, pasa a través de la córnea, la pupila y el cristalino, y llega por último a la retina, donde la energía electromagnética de la luz se convierte en impulsos nerviosos que pueden ser utilizados por el cerebro. Los impulsos abandonan el ojo a través del nervio
óptico. La región más sensible del ojo en la visión normal diurna es una pequeña depresión de la retina llamada fóvea en el cual se enfoca la luz que viene del centro del campo visual (por campo visual entendemos aquello a lo que mira el sujeto). Puesto que la lente simple convexa invierte la imagen, el campo visual derecho es representado a la izquierda de la retina y el campo inferior representado en lo alto de la retina. El ojo es un sistema óptico muy imperfecto. Las ondas de luz no solo tienen que pasar a través de los humores y el cristalino, después penetrar la red de los vasos sanguíneos y fibras nerviosas antes de que lleguen las células sensibles los bastones y los conos de la retina donde la luz se convierte en impulsos nerviosos. A pesar de estas imperfecciones el ojo funciona muy bien. La fóvea es capaz de percibir un cable telefónico a 400 m de distancia. En buenas condiciones el ojo puede percibir un alambre cuyo grosor no cubre más de 0,5 mm.

Tambien exister otras definiciones que indican que, el ojo es la puerta de entrada por la que ingresan los estímulos luminosos que se transforman en impulsos eléctricos gracias a unas células especializadas de la retina que son los conos y los bastones. Entonces, el nervio óptico transmite los impulsos eléctricos generados en la retina al cerebro, donde son procesados en la corteza visual. Finalmente, en el cerebro tiene lugar el complicado proceso de la percepción visual gracias al cual somos capaces de percibir la forma de los objetos, identificar distancias, detectar los colores y el movimiento \cite{14alonso2005personas}.


        \begin{figure}[H]
		\centering
		\includegraphics[width=120mm]{./Imagenes/estructura_percepcion.png}
		\caption{Estructura de la percepción visual humana.}
		\vspace{0.15cm}
		\textit{Fuente: Fernando Vila Arroyo, “El Libro Blanco de la Iluminación”. España 2013.}
		\label{fig:estructura_percepcion}
		\end{figure}      	  

\subsection{Visión por Computador}
La visión artificial o también conocida como visión por computador es una disciplina científica que incluye métodos para adquirir, procesar, analizar y comprender las imágenes del mundo real con el fin de producir información numérica o simbólica para que puedan ser tratados por un computador. Tal y como los humanos usamos nuestros ojos y cerebros para comprender el mundo que nos rodea, la visión por computador trata de producir el mismo efecto para que las computadoras puedan percibir y comprender una imagen o secuencia de imágenes y actuar según convenga en una determinada situación. Esta comprensión se consigue gracias a distintos campos como la geometría, la estadística, la física y otras disciplinas. La adquisición de los datos se consigue por varios medios como secuencias de imágenes, vistas desde varias cámaras de video o datos multidimensionales desde un
escáner médico.

Hay muchas tecnologías que utilizan la visión por computador(figura \ref{fig:esquema_vision_computador}), entre las cuáles tenemos: reconocimiento de objetos, detección de eventos, reconstrucción de una escena (\textit{mapping}) y restauración de imágenes \cite{15VC}.


\begin{figure}[H]
		\centering
		\includegraphics[width=120mm]{./Imagenes/esquema_vision_computador.png}
		\caption{Esquema de las relaciones entre la visión por computadora y otras áreas afines.}
		\vspace{0.15cm}
		\textit{Fuente: Propio}
		\label{fig:esquema_vision_computador}
\end{figure}  


\section{Deteccion de Rostros}
En los últimos años se ha hecho una gran cantidad de esfuerzo en el campo de la detección de rostros. La cara humana contiene características importantes que pueden ser utilizados por los sistemas automatizados basados en la visión con el fin de identificar y reconocer a los individuos. En la localización del rostro, la etapa primaria de los sistemas automatizados basados en la visión es encontrar el área de la cara en la imagen de
entrada. La ubicación exacta de la cara es todavía una tarea difícil. Viola-Jones ha sido ampliamente utilizada por los investigadores con el fin de detectar la ubicación de las caras y los objetos en una imagen dada. Clasificadores de detección de rostros son compartidos por las comunidades públicas, tales como OpenCV \cite{20padilla2012evaluation}.

\subsection{Haar Cascade}
El detector de cara Viola-Jones motivado por el desafío de la detección de rostros, propuso un \textit{framework} detector de objetos utilizando características de tipo \textit{Haar}, que ha sido ampliamente utilizado por otros trabajos, no sólo para la detección de rostros, sino también para la ubicación de objetos. Gracias a la implementación \textit{Open Computer Vision Library}(OpenCV), el framework general de detectores de objetos se ha popularizado y ha motivado a la comunidad a generar sus propios clasificadores de objetos. Estos clasificadores usan características parecidas a las del \textit{Haar} que se aplican sobre la imagen. Solamente aquellas regiones de imagen, llamadas sub-ventanas, que pasan a través de todas las etapas del detector, se considera que contienen el objeto objetivo. La figura \ref{fig:deteccion_cascade} muestra el esquema de cascada de detección con N etapas. La cascada de detección está diseñada para eliminar un gran número de ejemplos negativos con un poco de
procesamiento \cite{20padilla2012evaluation}.

\begin{figure}[H]
		\centering
		\includegraphics[width=100mm]{./Imagenes/deteccion_cascade.png}
		\caption{Detección Cascade.}
		\vspace{0.15cm}
		\textit{Fuente: Evaluation of Haar Cascade Classifiers for Face Detection.}
		\label{fig:deteccion_cascade}
\end{figure} 

\section{Redes Neuronales}
\subsection{Biológicas}

Son el principal elemento del Sistema Nervioso. Las redes neuronales biológicas son el resultado de la union de varias neuronas entrelazadas entre si. Una neurona es una célula compuesta por tres partes fundamentales: el cuerpo, un numero de extensiones llamadas dendritas que sirven de entradas, y una larga extensión llamada axón, la cual se activa como salida. Existe un proceso de comunicacion entre neuronas, el cual es conocido como 'la sinapsis', este proceso conecta el axón de una neurona a las dendritas de las otras neuronas para comunicarse por medio de impulsos electricos. Las neuronas están dispuestas en multiples capas. Por lo general las neuronas de una primera capa reciben entradas desde otra capa y envían sus salidas o impulsos nerviosos a las neuronas de una tercera. Existe un proceso de retroalimentación que se origina cuando los impulsos nerviosos de una neuronal son enviadas a ella misma, originando asi un ciclo donde la imformacion se mantiene por periodos de tiempo. Similar, puede ocurrir la comunicacion entre neuronas de la misma capa.

Las conexiones entre neuronas tienen pesos asociados que representan la influencia de una sobre la otra. Si dos neuronas no están conectadas, el correspondiente peso de enlace es cero. Esencialmente, cada una envía su información de estado multiplicado por el correspondiente peso a todas las neuronas conectadas con ella. Luego
cada una, a su vez, suma los valores recibidos desde sus dendritas para actualizar sus estados respectivos.

Se emplea normalmente un conjunto de ejemplos representativos de la
transformación deseada para "entrenar" el sistema, que, a su vez, se adapta para producir
las salidas deseadas cuando se lo evalúa con las entradas "aprendidas".

Además, se producirán respuestas cuando, en la utilización, se presenten entradas
totalmente nuevas para sistema, esto es durante el modo entrenamiento la información
sobre el sistema a resolver es almacenada dentro del ANN y la red utiliza su modo
productivo en ejecutar transformaciones y aprender. De este modo el sistema de red
neuronal no reside necesariamente en la elegancia de la solución particular sino en su
generalidad de hallar solución a problemas particulares, habiéndose proporcionado
ejemplos del comportamiento deseado. Esto permite la evolución de los sistemas
autómatas sin una reprogramación explicita.

Las redes neuronales artificiales se basan en el circuito de procesamiento de
entradas en el cual los pesos son sumados. Las funciones de peso serán llamadas desde
ahora como atenuadores. En la implementación, las entradas a una neurona son pesadas
multiplicando el valor de la entrada por un factor que es menor o igual a uno. El valor de
los factores de peso es determinado por el algoritmo de aprendizaje \cite{21RedesNeuronales}.

Las entradas atenuadas son sumadas usando una función no lineal llamada Función
"Sigmoide". Si la salida de la función suma excede el valor de entrada máximo de la
neurona, esta responde generando una salida.

Una persona tiene alrededor de $10^{11}$ neuronas, cada una con alrededor de $10^4$
salidas. La estructura de neuronas de la corteza cerebral es modular: si bien todas las
partes del cerebro son relativamente similares, diferentes partes hacen diferentes cosas; a
partir de una estructura general, según la experiencia se generan nuevas estructuras
especificas al problema a resolver \cite{16pusiol2014redes}. 


\begin{figure}[H]
		\centering
		\includegraphics[width=100mm]{./Imagenes/neurona_biologica.png}
		\caption{neuronal biológica}
		Source: Patri Tezanos, Neurociencia, 2016.
		\label{fig:neurona_biologica}
\end{figure} 


\subsection{Artificiales}
Las Redes Neuronales Artificiales (ANN) imitan su funcionamiento a aquellas
que se encuentran en el ámbito biológico. Son aptas para resolver problemas que no
poseen un algoritmo claramente definido para transformar una entrada en una salida;
aprenden, reconocen y aplican relaciones entre objetos.

Se emplea normalmente un conjunto de ejemplos representativos de la
transformación deseada para "entrenar" el sistema, que, a su vez, se adapta para producir
las salidas deseadas cuando se lo evalúa con las entradas "aprendidas".

Además, se producirán respuestas cuando, en la utilización, se presenten entradas
totalmente nuevas para sistema, esto es durante el modo entrenamiento la información
sobre el sistema a resolver es almacenada dentro del ANN y la red utiliza su modo
productivo en ejecutar transformaciones y aprender. De este modo el sistema de red
neuronal no reside necesariamente en la elegancia de la solución particular sino en su
generalidad de hallar solución a problemas particulares, habiéndose proporcionado
ejemplos del comportamiento deseado. Esto permite la evolución de los sistemas
autómatas sin una reprogramación explicita.

Las Redes Neuronales Artificiales se basan en el circuito de procesamiento de
entradas en el cual los pesos son sumados. Las funciones de peso serán llamadas desde
ahora como atenuadores. En la implementación, las entradas a una neurona son pesadas
multiplicando el valor de la entrada por un factor que es menor o igual a uno. El valor de
los factores de peso es determinado por el algoritmo de aprendizaje.

Las entradas atenuadas son sumadas usando una función no lineal llamada
Función "Sigmoide". Si la salida de la función suma excede el valor de entrada máximo
de la neurona, esta responde generando una salida.

Cada neurona tiene varias entradas y su salida está conectada a un conjunto de
otros procesadores de entradas.

Cuando una ANN funciona en modo normal, a partir de los datos presentados en
la entrada, se genera un patrón especifico de salida. La relación Entrada/Salida será
determinada durante el modo entrenamiento, entonces cuando una entrada conocida es
presentada da la salida esperada.

El algoritmo de entrenamiento ajusta los pesos de las entradas hasta que se alcanza
la salida esperada.

Las neuronas en la figura tienen una leve complejidad computacional, porque solo
se comunican con las neuronas más cercanas conectándose de forma simple. Por las
características y capacidades que ofrece la tecnología VLSI es posible (en costos)
construir una Red Neuronal con muchos procesadores \cite{21RedesNeuronales}.


\begin{figure}[H]
		\centering
		\includegraphics[width=100mm]{./Imagenes/neurona_artificial.png}
		\caption{Modelo matemático de una red neuronal}
		Source: Yuly Cristina Moreira Monserrate, Inteligencia Artificial, 2015.
		\label{fig:neurona_artificial}
\end{figure}

\begin{equation}\label{eq:funcion_salida}
Y_{i} = f(\sum W_{i,j}X{j} - \theta{i})
\end{equation}
Equation \eqref{eq:funcion_salida} Función de salida de una neurona artificial.


\section{ARQUITECTURA DE UNA RED NEURONAL ARTIFICIAL}
\subsection{Capas}

\begin{figure}[H]
		\centering
		\includegraphics[width=100mm]{./Imagenes/capas_red_neuronal.png}
		\caption{Capas de una red neuronal artificial}
		Source: Propio
		\label{fig:capa_red_neuronal}
\end{figure} 

Una red neuronal se compone de tres capas:
\begin{itemize}
\item \textbf{Capa de Entrada.- }Es la capa que recibe cada uno de los números de la
lista de números entrante correspondientes a la matriz que representa una
imagen.
\item \textbf{Capa Oculta.- }Esta capa contiene unidades no observables, recibe
información de la capa entrante, para posteriormente procesarla y manda
información a la capa de salida.
\item \textbf{Capa de Salida.-} Contiene los resultados como una lista de números.
\end{itemize}


\subsection{Funciones de Activación}
La función de activación recibe como entrada la suma de todos los números que
llegan por las conexiones entrantes, transforma el valor mediante una fórmula, y produce
un nuevo número. Existen varias opciones. Uno de los objetivos de la función de
activación es mantener los números producidos por cada neurona dentro de un rango
razonable (por ejemplo, números reales entre 0 y 1).

\begin{itemize}
\item \textbf{Función de activación Sigmoide}\\

Muchos procesos naturales y curvas de aprendizaje de sistemas complejos
muestran una progresión temporal desde unos niveles bajos al inicio, hasta
acercarse a un clímax transcurrido un cierto tiempo; la transición se produce en
una región caracterizada por una fuerte aceleración intermedia. La función
Sigmoide permite describir esta evolución. Su gráfica tiene una típica forma de
"S". A menudo la función Sigmoide se refiere al caso particular de la función
logística y que viene definida por la siguiente ecuación \cite{24fsigmoide}:

\begin{equation}\label{eq:funcion_sigmoide}
f(x) = \frac{1}{1+\exp^{-x}}
\end{equation}
Equation \eqref{eq:funcion_sigmoide} Función Sigmoide.

\item \textbf{Función de activación Tangencial} \\

Es la versión continua de la función signo y se usa en problemas de
aproximación. Es importante por sus propiedades analíticas. Es continua a valores
en [-1,1] e infinitamente diferenciable, Esta función está definida como \cite{25ftangencial}

\begin{equation}\label{eq:funcion_tanh}
tanh(x) = \frac{\exp^x - \exp^{-x}}{\exp^x + \exp^{-x}}
\end{equation}
Equation \eqref{eq:funcion_tanh} Función Tangencial.


\item \textbf{Función de activación RELU (Rectified Linear Unit)} \\

Se conoce como una función de rampa y es análoga a la rectificación de
onda media en la ingeniería eléctrica. Esta función de activación fue introducida
por primera vez a una red dinámica por Hahnloser et al, en un artículo de año
2000, con fuertes motivaciones biológicas y justificaciones matemáticas. Se ha
utilizado en las Redes Convolucionales con más eficacia que el ampliamente
utilizado Sigmoide logística (que se inspira en la teoría de probabilidades) y su
más práctico contraparte, la tangente hiperbólica . El rectificador es a partir del
2015, la función de activación más popular para las Redes Neuronales Profundas 
\cite{26fRelu}.

\begin{equation}\label{eq:funcion_Relu}
f(x)=Max(0,x)
\end{equation}
Equation \eqref{eq:funcion_Relu} Función RELU.

\end{itemize}
\subsection{Bias o Sesgo}
Justo antes de aplicar la función de activación, cada neurona añade a la suma de
productos un nuevo término constante, llamado habitualmente bias, cuyo único objetivo
es lograr una convergencia más rápida de la red.


\begin{figure}[H]
		\centering
		\includegraphics[width=100mm]{./Imagenes/arquitectura_red_neuronal.png}
		\caption{Arquitectura de un RNA incluida el sesgo}
		Source: Propio
		\label{fig:arquitectura_red_neuronal}
\end{figure}

\section{IMPLEMENTACIÓN DE UNA RNA}
Una forma sencilla de implementar redes de neuronas consiste en almacenar los
pesos en matrices. Posteriormente guardar los valores de todas las neuronas de la capa en
un vector, el producto del vector y la matriz de pesos de salida, nos da los valores de
entrada de cada neurona en la siguiente capa. Después se aplica la función de activación
que hayamos elegido a cada elemento de ese segundo vector, y repetir el proceso.

\section{BACKPROPAGATION}
El BackPropagation es un algoritmo de aprendizaje supervisado que se usa para entrenar
redes neuronales artificial, dicho algoritmo se basa en el descenso de gradiente que es un
algoritmo de optimización utilizado para determinar los valores de los parámetros
(coeficientes) de una función (f) que minimiza una función de costes. El descenso de
gradiente se utiliza mejor cuando los parámetros no pueden ser calculados analíticamente
(por ejemplo, usando algebra lineal) y deben ser buscados por un algoritmo de
optimización \cite{27lehr1993backpropagation}.


\begin{figure}[H]
		\centering
		\includegraphics[width=100mm]{./Imagenes/back_propagation.png}
		\caption{Descenso de gradiente}
		Source: Propio
		\label{fig:back_propagation}
\end{figure}


\section{DEEP LEARNING}
El Deep Learning es un concepto muy amplio, lo que conlleva a que no tenga solo
una definición veraz. Sin embargo, se puede generalizar en que el Deep Learning es un
concepto que surge de la idea de imitar el celebro a partir del uso de hardware y software,
para crear una inteligencia artificial pura, utilizando una capacidad de abstracción
jerárquica, es decir, una representación de los datos de entrada en varios “niveles”, en el
caso de las RNA, en varias capas, para seleccionar características que son útiles para el
aprendizaje; de esta manera, una característica de un nivel de complejidad más alto será
aprendido de una de un nivel de complejidad más bajo.

El Deep Learning es un conjunto de algoritmos en Machine Learning que intenta
modelar abstracciones de alto nivel en datos usando arquitecturas compuestas de
transformaciones no-lineales múltiples \cite{17bengio2013representation}.

Dependiendo de la RNA, el algoritmo de entrenamiento de las RNA “más
simples”, de las cuales están compuestas las arquitecturas profundas, se pueden
caracterizar, principalmente, en dos categorías:

\begin{itemize}
\item \textbf{Supervisado: } Se caracteriza porque su entrenamiento es controlado por un agente
externo. Este agente externo “guía” el entrenamiento de la red mediante una
comparación entre las salidas deseadas y las salidas que proporciona la red \cite{18restrepo2015aplicacion}.

\begin{figure}[H]
		\centering
		\includegraphics[width=75mm]{./Imagenes/grafico_supervisado.png}
		\caption{Aprendizaje supervisado}
		Source: LÓPEZ S, Jesús A. CAICEDO B, Eduardo F.
		\label{fig:grafico_supervisado}
\end{figure}



\item \textbf{No supervisado: }
El aprendizaje es realizado presentándole a la red los datos
directamente, es decir, ahora no existe un agente supervisando en el
entrenamiento, la red aprende los datos de la entrada modificando los pesos en
función de los datos caracterizados formando, en algunos casos, clusters o
agrupación de los datos, tendiendo a clasificar los datos de forma probabilística
\cite{18restrepo2015aplicacion}.

\begin{figure}[H]
		\centering
		\includegraphics[width=75mm]{./Imagenes/grafico_no_supervisado.png}
		\caption{Aprendizaje no supervisado}
		Source: LÓPEZ S, Jesús A. CAICEDO B, Eduardo F.
		\label{fig:grafico_no_supervisado}
\end{figure}

\item \textbf{Híbrido: }
En las arquitecturas del Deep Learning, algunas redes poseen o utilizan
ambos tipos de entrenamientos, ya sea comenzando con un pre-entrenamiento
supervisado y finalizando con uno no supervisado o viceversa. Esto es con el fin
de lograr un ajuste fino, disminuir el tiempo de convergencia, entre otras
funcionalidades \cite{18restrepo2015aplicacion}.
\end{itemize}
\section{MODELOS MÁS COMUNES DEL DEEP \\LEARNING}
\subsection{Autoencoder}
Es una Red Neuronal Artificial utilizada para el aprendizaje no supervisado
de codificaciones eficientes . El objetivo de una autoencoder es aprender una
representación (codificación) para un conjunto de datos, típicamente con el propósito
de reducción de dimensionalidad . Recientemente, el concepto autoencoder se ha vuelto
más ampliamente utilizado para el aprendizaje de modelos generativos de datos \cite{28fAutoencoder}.

Un auto-codificador, o autoencoder, aprende a producir a la salida exactamente la
misma información que recibe a la entrada. Por eso, las capas de entrada y salida siempre
deben tener el mismo número de neuronas. Por ejemplo, si la capa de entrada recibe los
píxeles de una imagen, esperamos que la red aprenda a producir en su capa de salida
exactamente la misma imagen que ha sido introducido \cite{18restrepo2015aplicacion}.

\begin{figure}[H]
		\centering
		\includegraphics[width=100mm]{./Imagenes/autoenconder.png}
		\caption{Arquitectura de una red neuronal Auto-encoder}
		Source: Propio
		\label{fig:autoencoder}
\end{figure}

\subsection{Redes Neuronales Recurrentes}
Las Redes de Neuronas Recurrentes (Recurrent Neural Networks) no tienen una
estructura de capas, sino que permiten conexiones arbitrarias entre todas las neuronas,
incluso creando ciclos. Esto permite incorporar a la red el concepto de temporalidad, y
permite que la red tenga memoria, porque los números que introducimos en un momento
dado en las neuronas de entrada son transformados, y continúan circulando por la red
incluso después de cambiar los números de entrada por otros diferentes \cite{18restrepo2015aplicacion}.


\begin{figure}[H]
		\centering
		\includegraphics[width=100mm]{./Imagenes/red_recurrente.png}
		\caption{Arquitectura de una red neuronal Recurrente.}
		Source: Propio
		\label{fig:red_recurrente}
\end{figure}

\subsection{Redes Neuronales Convolucionales}
Las Redes Neuronales Convolucionales (Convolution Neural Network) mantienen
el concepto de capas, pero cada neurona de una capa no recibe conexiones entrantes de
todas las neuronas de la capa anterior, sino sólo de algunas. Esto favorece que una neurona
se especialice en una región de la lista de números de la capa anterior, y reduce
drásticamente el número de pesos y de multiplicaciones necesarias. Lo habitual es que
dos neuronas consecutivas de una capa intermedia se especialicen en regiones solapadas
de la capa anterior \cite{16pusiol2014redes}.

\section{ARQUITECTURA DE UNA RED NEURONAL
CONVOLUCIONAL}

Las Redes Neuronales Convolucionales es una estructura compuesta de varias
fases entrenables, aprendiendo de cada una de las características con diferentes grados de
abstracción. La entrada y salida de cada una de estas etapas son conjunto de arreglos
llamados mapas de características, a la salida cada mapa de características representa una
característica particular extraída de la imagen de entrada.

Cada fase está compuesta por tres capas: Convolucion, función no lineal y una
capa de sub-muestreo.

Una típica arquitectura de Red Neuronal Convolucional para clasificación
supervisada está basada en varias etapas seguidas de un clasificador, por ejemplo, la red
de Yann LeCun para resolver el problema de reconocimiento de caracteres, utilizo una
arquitectura con dos fases.


\begin{figure}[H]
		\centering
		\includegraphics[width=100mm]{./Imagenes/arquitectura_CNN_Lecun.png}
		\caption{Arquitectura de una red neuronal Convolucional.}
		Source: Yann LeCun, 1998.
		\label{fig:arquitectura_CNN_Lecun}
\end{figure}

Viendo el funcionamiento de la arquitectura del Let-Net (Figura 13) toma como
entrada una imagen, y en la entrada de la primera fase hay una secuencia de mapas de
características producto de la convolucion, seguida por una capa de sub-muestreo. En la
capa de convolucion cada uno de los seis mapas de características contiene pequeñas
características con sus pesos agrupados. A continuación, se encuentra la capa del sub-
muestreo que agrupa las salidas de una serie de características replicadas vecinas en la
capa de convolución, dando como resultado un mapa de características más pequeño que
servirá de entrada para la siguiente fase dedicada a encontrar características replicadas de
mayor abstracción. A medida que se avanza en las fases se aprenden características más
complicadas, pero más invariantes a posición (Por el sub-muestreo).

Las capas enteramente conectadas se encargan de evaluar las posibles
combinaciones de las características aprendidas para lograr clasificar las imágenes dadas
\cite{16pusiol2014redes}.

\subsection{Capa de Convolución}
La capa de convolución es el bloque de construcción básico de una red de
convolución que hace la mayor parte del trabajo pesado computacional.
\begin{itemize}
\item \textbf{Visión general e intuición sin cerebro.} La capa de convolución calcula sin
analogías (cerebro / neurona). La capa de parámetros de convolución consisten en
un conjunto de filtros que se pueden aprender. Cada filtro es pequeño
espacialmente (a lo largo de la anchura y altura), sino que se extiende a través de
toda la profundidad del volumen de entrada. Por ejemplo, un filtro típico en una
primera capa de una Red Neuronal Convolucional podría tener un tamaño de
5x5x3 (es decir, 5 píxeles anchura y la altura, y 3 ya que las imágenes tienen
profundidad 3, los canales de color). Durante el pase hacia adelante, se desliza
(más precisamente, convolución) cada filtro a través del ancho y la altura del
volumen de entrada y calcular productos escalares entre las entradas del filtro y la
entrada en cualquier posición. A medida que se desplaza el filtro sobre la anchura
y la altura del volumen de entrada se produce un mapa de activación de 2
dimensiones que da las respuestas de ese filtro en cada posición
espacial. Intuitivamente, la red aprenderá filtros que se activan cuando ven algún
tipo de función visual, como un borde de una orientación o una mancha de un cierto color en la primera capa, o patrones de panal. Después se tendrá todo un
conjunto de filtros en cada capa de convolución (por ejemplo, 12 filtros), y cada
uno de ellos va a producir un mapa de activación de 2 dimensiones por
separado. Vamos a apilar estos mapas de activación a lo largo de la dimensión de
la profundidad y producir el volumen de salida.

\item \textbf{La vista del cerebro.} Cada entrada en el volumen de salida 3D también se puede
interpretar como una salida de una neurona que mira sólo una pequeña región en
los parámetros de entrada y comparte con todas las neuronas a la izquierda y
derecho espacial (ya que todos estos números resultaría de aplicar el mismo
filtro).

\item \textbf{Conectividad local.} Cuando se trata de entradas de alta dimensión como las
imágenes, como se vio anteriormente, no es práctico conectar neuronas a todas las
neuronas en el volumen anterior. En su lugar, se va a conectar cada neurona a sólo
una región local del volumen de entrada. La extensión espacial de esta
conectividad es un hiperparámetro llamado campo receptivo de la neurona
(equivalentemente este es el tamaño del filtro). La extensión de la conectividad a
lo largo del eje de profundidad es siempre igual a la profundidad del volumen de
entrada. Es importante destacar nuevamente esta asimetría en cómo tratamos las
dimensiones espaciales (anchura y altura) y la dimensión de la profundidad: Las
conexiones son locales en el espacio (a lo largo del ancho y la altura), pero siempre
llenas a lo largo de toda la profundidad del volumen de entrada \cite{22RedesNeuronalesConvolu}.


\begin{figure}[H]
		\centering
		\includegraphics[width=70mm]{./Imagenes/convolucion.png}
		\caption{Ejemplo de convolución con una ventana de 2X2}
		Source: Rubén López, https://rubenlopezg.wordpress.com/2014/05/07/que-es-y-como-funciona-
deep-learning/
        \label{fig:convolucion}
\end{figure}


\item { \textbf{Pseudo – Codigo.} Los valores de un píxel dado en la imagen de salida se calculan
multiplicando cada valor del kernel por los valores de píxeles de la imagen de
entrada correspondientes. Esto se puede describir algorítmicamente con el
siguiente pseudo-código:

%\vspace{6cm}

\begin{algorithm}
\caption{Pseudo-Codigo Convolucion\\
La convolucion de una image f(x,y) con un kernel k(x,y) con dimensiones HxW y (2h+1)x(2w+1) respectivamente produce una nueva imagen g(x,y)}\label{alg:euclid}
\begin{algorithmic}[H]
\Procedure{Convolucion}{$f,k$}\Comment{La convolucion de la imagen f con el kernel k}

\For{\texttt{y:=1 to W}}

\For{\texttt{x:=1 to H}}
  \State $sum=0$
     \For{\texttt{i:=-h to h}}
		\For{\texttt{j:=-w to w}}
           \State $sum=sum + k(j,i)*f(x-j,y-i)$
		\EndFor
	 \EndFor 
  \State $g(x,y)=sum$
\EndFor
\EndFor
\State \textbf{return} $g$\Comment{El resultado de la convolucion entre f y k}
\EndProcedure
\end{algorithmic}
\end{algorithm}


\begin{equation}\label{eq:ecuac_conv}
V = \frac{\sum^{q1}_{i=0}\sum^{q2}_{i=j} f_{i,j}*k_{i,j}}{F}
\end{equation}
Equation \eqref{eq:ecuac_conv} Formula de Convolución.
\\
Donde:\\
\begin{itemize}
\item $f_{i,j}$: El pixel en la posición $i,j$ de la imagen f respecto al kernel k.
\item $k_{i,j}$: El pixel en la posición $i,j$ del kernel k.
\item $q1xq2 = (2h+1)x(2w+1)$: La dimension del kernel.
\item F: La suma de los coeficientes del kernel, o 1 si la suma es igual a 0.
\item g(i,j): El valor de salida de un pixel.
\end{itemize}}

\end{itemize}


\subsection{Submuestreo}
Es común insertar periódicamente una capa de agrupación entre capas sucesivas
de convolución en una arquitectura de Red Neuronal Convolucional. Su función es
reducir progresivamente el tamaño espacial de la representación para reducir la cantidad
de parámetros y el cálculo en la red, y por lo tanto también para controlar el sobre ajuste.
La capa de agrupación funciona independientemente en cada segmento de profundidad
de la entrada y la redimensiona espacialmente, utilizando la operación MAX. La forma
más común es una capa de agrupación con filtros de tamaño 2x2 aplicado con una zancada
de 2 muestras descendentes cada porción de profundidad en la entrada por 2 a lo largo
tanto de ancho como de altura, descartando el 75\% de las activaciones. En este caso, cada
operación MAX tomaría un máximo de 4 números (pequeña región 2x2 en una parte de
profundidad). La dimensión de profundidad no cambia. Más generalmente, la capa de
agrupación:

\begin{itemize}
\item Acepta un volumen de tamaño W1xH1xD1
\item Requiere 2 hiperparámetro
  \begin{itemize}
  \item Su extensión espacial F
  \item La zancada S
  \end{itemize}
\item Produce un volumen de tamaño: W2xH2xD2
  \begin{itemize}
  \item $W2 =  \frac{W1 - F}{S + 1}$
  \item $H2 = \frac{H1 - F}{S + 1}$
  
  \item D2 = D1
  \end{itemize}
\item Introduce parámetros cero, ya que calcula una función fija de la entrada.
\item Tiene en cuenta que no es común utilizar cero como relleno para las capas de
agrupación
\end{itemize}

Sólo hay dos variaciones comunes de la capa de agrupación máxima encontrada
en la práctica: Una capa de agrupación con F = 3, S = 2F, S = 2 (también llamada
superposición de agrupación) y más comúnmente F = 2, S = 2F, S = 2 Los tamaños
de agrupación con campos receptivos más grandes son demasiado destructivos \cite{22RedesNeuronalesConvolu}.

\begin{figure}[H]
		\centering
		\includegraphics[width=100mm]{./Imagenes/submuestre.png}
		\caption{Ejemplo de Submuestreo con una ventana de 2X2 y calculando el promedio}
		Source: Rubén López, https://rubenlopezg.wordpress.com/2014/05/07/que-es-y-como-funciona-
deep-learning/
        \label{fig:submuestre}
\end{figure}


\subsection{Capa de normalización}
Normalizar las activaciones de la capa anterior en cada lote, es decir, se aplica una
transformación que mantiene la activación de cierre media de 0 y la desviación estándar
de activación cerca de 1 \cite{22RedesNeuronalesConvolu}.

\subsection{Capa totalmente conectada}
Las neuronas en una capa completamente conectada tienen conexiones completas con
todas las activaciones en la capa anterior, como se ve en las redes neuronales regulares.
Por tanto, sus activaciones pueden calcularse con una multiplicación matricial seguida de
un desplazamiento de polarización \cite{22RedesNeuronalesConvolu}.


\begin{figure}[H]
		\centering
		\includegraphics[width=100mm]{./Imagenes/grafico_full_conect.png}
		\caption{Capa totalmente conectada}
		Source: Michael A. Nielsen, http://neuralnetworksanddeeplearning.com/chap6.html
		\label{fig:grafico_full_conect}
\end{figure}

\subsection{Función de normalización(Softmax)}
La regresión softmax es sólo otro nombre para regresión lineal multinomial o
simplemente clase múltiple de regresión logística.

En su esencia, regresión de softmax es una generalización de la regresión logística
que podemos utilizar para la clasificación de clase múltiple (bajo el supuesto de que las
clases son mutuamente excluyentes). En cambio, utilizamos el modelo de regresión
logística (estándar) en tareas de clasificación binario.

\begin{figure}[H]
		\centering
		\includegraphics[width=100mm]{./Imagenes/arquitectura_cnn_softmax.png}
		\caption{Arquitectura de una CNN con Softmax}
		Source: Samuel Salvatella, http://ssalva.bitballoon.com/blog/2016-08-30-tensorflow/
		\label{fig:arquitectura_cnn_softmax}
\end{figure}


En las matemáticas , la función softmax , o función exponencial normalizada , es
una generalización de la función logística que permite la utilización de un vector de
dimensión.. La función está dada por

\begin{equation}\label{eq:ecuac_soft}
P(Y = j|Z^i) = \phi_{softmax}(Z^i)= \frac{\exp^{Z^i}}{\sum^k_{j=0}\exp^{Z^{i}_k}}
\end{equation}
Equation \eqref{eq:ecuac_soft} Formula softmax, Donde:\\
$Z = w_{0}x_{0} + w_{1}x_{1} +...+w_{m}x_{m} = \sum_{l=0}^m w_{l}x_{l} = w^Tx$


\section{ENTRENAMIENTO DE UNA RED NEURONAL CONVOLUCIONAL}

El proceso de la CNN para la parte del entrenamiento utiliza el algoritmo
BackPropagation que consiste en calcular una función objetivo que es el de minimizar el
error haciendo para esto la retro propagación del error obtenido a las capas anteriores a la
salida para que se ajusten los pesos de las conexiones entre neuronas.

El algoritmo BackPropagation trabaja de la siguiente forma:

\begin{itemize}
\item Se dan datos de entrada a la red neuronal.
\item Propaga dichas entradas hasta la capa de salida con pesos iniciales definidos o
aleatorios.
\item Calcula el error en la capa de salida.
\item Propaga dicho error hacia las neuronas ocultas (hacia atrás).
\item Cambia los pesos de las conexiones.
\end{itemize}

\section{SOBRE LAS EXPRESIONES FACIALES}
\subsection{Paul Ekman}

Después de que su madre desarrolló una enfermedad mental y se suicidó, Paul
Ekman (psicólogo y científico del comportamiento) dedicó su vida a la Psicoterapia y
ayudar a las personas con trastornos mentales. Él comenzó su investigación en la
comunicación no verbal en la década de 1950, el desarrollo de maneras sistemáticas para
medir el lenguaje corporal. En el proceso, descubrió que, a través de la investigación
empírica, pudo identificar constantemente las expresiones faciales creadas por el
movimiento de los músculos de la cara. Y así, Ekman amplió su investigación para incluir
expresiones faciales y sus significados\cite{29ekman2016scientists}.


\subsection{Las seis emociones básicas}
Antes de Ekman llegó a la escena, se creía ampliamente (por antropólogos
incluyendo Margaret Mead) que las expresiones faciales y las emociones que ellos
representan se determinaron por la cultura – que las personas aprendieron a hacer y leer
las expresiones faciales de sus sociedades. Ekman se dispuso a probar esta idea en
1968. Él viajó a Papúa Nueva Guinea para estudiar las expresiones faciales de los
miembros de la tribu Fore apartada, donde aprendió que podían identificar
constantemente las emociones en las expresiones faciales por mirar fotos de la gente de
otras culturas, a pesar de que la tribu no había sido expuesta a cualquier exterior culturas.

Se hizo evidente, entonces, que las expresiones faciales son interculturales, su
investigación reveló que existe un conjunto universal de ciertas expresiones faciales se
utilizan tanto en el mundo occidental y oriental. Esta lista de expresiones faciales
universales, que Ekman publicó en el año 1972, dispone de las seis emociones
básicas. Tomar por lo vistazo a la lista, así como imágenes, definiciones y movimientos
musculares de estas emociones, a continuación:

\begin{itemize}
\item {\textbf{Cólera:} 
\begin{itemize}
\item \textbf{Descripción.-} El antagonismo hacia una persona o un objeto a menudo se sentía
después de que usted siente que ha sido agraviado u ofendido.
\item { \textbf{Movimientos musculares faciales.-} La reducción de las cejas, apretar y estrechar
los labios, los ojos mirando, apretando los párpados inferiores, con menos
frecuencia, empujando la mandíbula hacia adelante.

\begin{figure}[H]
		\centering
		\includegraphics[width=20mm]{./Imagenes/colera.png}
		\caption{Expresión Facial de Cólera}
		Source: Paul Ekman, http://www.serperuano.com/2014/03/paul-ekman-las-6-emociones-basicas/
		\label{fig:colera}
\end{figure}}
\end{itemize}}



\item {\textbf{Felicidad:} 
\begin{itemize}
\item \textbf{Descripción.-} Agradable sensación de satisfacción y bienestar.
\item { \textbf{Movimientos musculares faciales.-} Smiling – tirando hacia arriba comisuras de
la boca, contrayendo los músculos grandes orbitales alrededor de los ojos.

\begin{figure}[H]
		\centering
		\includegraphics[width=20mm]{./Imagenes/felicidad.png}
		\caption{Expresión Facial de Felicidad}
		Source: Paul Ekman, http://www.serperuano.com/2014/03/paul-ekman-las-6-emociones-basicas/
		\label{fig:felicidad}
\end{figure}}
\end{itemize}}



\item {\textbf{Sorpresa:} 
\begin{itemize}
\item \textbf{Descripción.-} Sensación de malestar o sorpresa ante un hecho inesperado.
\item { \textbf{Movimientos musculares faciales.-} Levantando las cejas altas (que puede causar
arrugas en la frente), abriendo los ojos como platos, dejando caer la mandíbula
tan boca es ágape.

\begin{figure}[H]
		\centering
		\includegraphics[width=20mm]{./Imagenes/sorpresa.png}
		\caption{Expresión Facial de Sorpresa}
		Source: Paul Ekman, http://www.serperuano.com/2014/03/paul-ekman-las-6-emociones-basicas/
		\label{fig:sorpresa}
\end{figure}}
\end{itemize}}


\item {\textbf{Asco:} 
\begin{itemize}
\item \textbf{Descripción.-} Desagrado intenso o condena causada por algo ofensivo o
repulsiva.
\item { \textbf{Movimientos musculares faciales.-} La reducción de las cejas, curvando el labio
superior, arrugando la nariz.

\begin{figure}[H]
		\centering
		\includegraphics[width=20mm]{./Imagenes/asco.png}
		\caption{Expresión Facial de Asco}
		Source: Paul Ekman, http://www.serperuano.com/2014/03/paul-ekman-las-6-emociones-basicas/
		\label{fig:asco}
\end{figure}}
\end{itemize}}



\item {\textbf{Tristeza:} 
\begin{itemize}
\item \textbf{Descripción.-} Sentimiento de infelicidad o tristeza.
\item { \textbf{Movimientos musculares faciales.-} Los párpados caídos, la reducción de las
esquinas de la boca, labios fruncidos, los ojos bajos.

\begin{figure}[H]
		\centering
		\includegraphics[width=20mm]{./Imagenes/tristeza.png}
		\caption{Expresión Facial de Tristeza}
		Source: Paul Ekman, http://www.serperuano.com/2014/03/paul-ekman-las-6-emociones-basicas/
		\label{fig:tristeza}
\end{figure}}
\end{itemize}}


\item {\textbf{Miedo:} 
\begin{itemize}
\item \textbf{Descripción.-} Sensación de aprehensión provocada por la percepción de peligro,
amenaza o imposición de dolor.
\item { \textbf{Movimientos musculares faciales.-} Levantando las cejas / dibujar las cejas
juntas, tensando los párpados inferiores, que se extiende horizontalmente labios,
la boca ligeramente abierta.

\begin{figure}[H]
		\centering
		\includegraphics[width=20mm]{./Imagenes/miedo.png}
		\caption{Expresión Facial de Miedo}
		Source: Paul Ekman, http://www.serperuano.com/2014/03/paul-ekman-las-6-emociones-basicas/
		\label{fig:miedo}
\end{figure}}
\end{itemize}}

\end{itemize}


\subsection{Otras expresiones faciales}
Los hallazgos de Ekman sobre las expresiones faciales universales revelaron el
carácter intercultural de la relación entre la comunicación no verbal y la emoción, sin
embargo, las teorías de Ekman han evolucionado desde que ideó su lista de emociones
básicas. En la década de 1990, añadió una serie de otros a la lista de emociones
universales, aunque hizo hincapié en que no todos ellos pueden ser identificados
utilizando expresiones faciales. Estas emociones adicionales son \cite{29ekman2016scientists}

\begin{itemize}
\item Diversión
\item Desprecio
\item Contentamiento
\item Vergüenza	
\item Emoción
\item Culpa
\item El orgullo de los logros
\item Alivio
\item Satisfacción
\item Placer sensorial
\item Vergüenza
\item Neutro
\end{itemize}


\part{Desarrollo del Proyecto}
\chapter{Desarrollo del Detector de Rostros y la Arquitectura de Red Neuronal Convolucional}

\section{Detección de Rostros}
En este trabajo se optó por utilizar como etapa inicial dentro de la fase de consultas a la red, la construcción de un detector de rostros, debido a que, al momento de desarrollar una aplicación orientada a las necesidades del mundo real, este no solo recibira como entrada imágenes que contengan exactamente el rostro de la persona, sino, imágenes con el cuerpo completo o algunas partes adicionales aparte del rostro. Sin embargo, en objetivo del trabajo es poder detectar la expresión facial de una persona, para lo cual, basta con tener como entrada a la red una imagen que delimite el rostro de la persona. De ahí, la necesidad de utilizar un algoritmo de detección de rostros para la extracción de la región de interes que posteriormente servira como entrada para la red neuronal convolucional encargada de reconocer la expresión facial correspondiente. La figura~\ref{fig:imagen_entrada} muestra un ejemplo de una imagen de entrada, en la cual, se puede observar detalles adicionales aparte del rostro(el sombrero y el fondo), los cuales no aportan carasteristicas relevantes que ayuden al reconocimiento de la expresion facial.

Para esta etapa se utilizó el detector de objetos \textit{Haar Cascade}, un algoritmo muy utilizado, cuya implementación puede ser encontrado en distintas librerías orientadas al procesamiento de imágenes, tales como OpenCV\footnote[5]{OpenCV es una librería \textit{open source} que contiene algoritmo relacionados con el area de visión por computador, http://opencv.org/}. Como se describió en la sección~\ref{sec:Haar_Cascade}, este algoritmo utiliza técnicas de \textit{machine learning}. Su proceso de entrenamiento se realiza con imágenes positivas y negativas(imágenes que representan y no representan rostros), creando así un modelo capaz de detectar rostros, basandose en la detecciín de caracteristicas \textit{Haar}. La entrada para esta etapa es una imagen cualquiera, el proceso consiste en detectar el rostro en dicha imagen(en caso exista algun rostro) y extraerlo en otra imagen en escala de grises, la cual tendra un tamaño aproximado de 48x48 pixeles(dependiendo de las dimensiones del rostro). Esta última imagen será la entrada para el modelo en la fase de consultas. Nótese que para la detección de un rostro y la asignación de su respectiva expresión facial, no es necesario mantener la imagen a coloros, puesto que este no es una característica necesaria para conseguir el objetivo. La figura~\ref{fig:proceso_deteccion} muestra los pasos a seguir para la detección y extracción del rostro en una imagen.

\begin{figure}[H]
		\centering
		\includegraphics[width=50mm]{Imagenes/imagen_entrada.png}
		\caption{Ejemplo de una imagen de entrada.}
		\vspace{0.15cm}
		\textit{Fuente: Consuelo Ferrús, http://www.acompasando.org/orar-el-asombro/}
		\label{fig:imagen_entrada}
\end{figure}


\begin{figure}[H]
		\centering
		\includegraphics[width=140mm]{Imagenes/proceso_deteccion.png}
		\caption{Proceso de detección de rostro.}
		\vspace{0.15cm}
		\textit{Fuente: Propio}
		\label{fig:proceso_deteccion}
\end{figure}

Se optó por la utilización del detector de rostros \textit{Haar Cascade}, debido a que este es ampliamente utilizado por el nivel de precisión que posee~\cite{6russakovsky2015imagenet}. Sin embargo, esta técnica aun presenta algunas fallas cuando el rostro presenta algún tipo de oclusión(figura~\ref{fig:oclussion}). Este tipo de problema tambien puede ser solucionado construyendo una red convolucional orientada a la detección o localización de rostros, o por medio de otro tipo de técnicas tradicionales basadas en la extracción de características. Debido al tiempo y bajos recursos computacionales, se opto por proponer este tipo de enfoques como trabajos futuros.

\begin{figure}[H]
		\centering
		\includegraphics[width=130mm]{Imagenes/oclussion.jpeg}
		\caption{Ejemplo de rostros con oclusión.}
		\vspace{0.15cm}
		\textit{Fuente: Face Dataset, Universidad Politécnica de Catalunya.}
		\label{fig:oclussion}
\end{figure}

\section{Experimentación en la Elección de Parámetros y Capas en la Construcción de la Arquitectura CNN}
\label{sec:experiment}
La creación de un modelo robusto a partir de la utilización de redes neuronales convolucionales, es una tarea muy complicada, debido a que en la actualidad no existen estudios que indique cual es la configuración correcta que esta debe seguir. Es así, que todos los trabajos que utilizan cualquier técnica perteneciente a \textit{deep learning}, se basan en la experimentación sobre diversar configuraciones en la arquitectura, estas configuraciones estan relacionadas con el numero de capas de convolución, sub-muestreo, totalmente conectadas, tipos de funciones de activacion y normalizacion. Otros elementos muy importantes que tambien son considerados al momento de realizar experimentos, son los parámetros de cada capa, tales como: el número y tamaño de los filtros en la capa de convolucion, operaciones de agrupación en la capa de sub-muestreo, etc.

Sin embargo, entrenar un red neuronal no es una tarea fácil, debido a que se tienen que optimizar miles de parametros para la creación de un buen modelo, asi como las miles de multiplicaciones de matrices que tienen que llevarse a cabo para realizar la operación de convolución. Es así que el número de experimentos que se puedan realizar, depende mucho de la insfraestructura y \textit{hardware} sobre el cual se esta trabajando, siendo este una gran limitante para una extensa experimentación.

Según lo explicado en los parrafos anteriores, para encontrar la correcta configuracion de una arquitectura basada en las redes neuronales convolucionales que resuelva el problema de reconocimiento de expresiones faciales, se realizaron muchos experimentos, cada uno con una configuracion diferente que la anterior. En total se evaluaron 3 arquitecturas con distintas configuracion de capas y parametros, en cada una de estas se calculo la precision y error utilizando las bases de datos FER2013 y CK.




\begin{itemize}
\item {\textbf{Conv-Conv-Pool-Conv-Conv-Pool-FC}

\begin{table}[H]
    \centering
    \includegraphics[width=140mm]{Imagenes/tabla_arqui_1_fer.png} 
    \caption{Evaluación de la arquitectura 1 y sus parámetros, FER2013}
    \label{tab:tabla_arqui_1_fer}
\end{table}

\begin{table}[H]
    \centering
    \includegraphics[width=140mm]{Imagenes/tabla_arqui_1_CK.png}
    \caption{Evaluación de la arquitectura 1 y sus parámetros, CK+}
    \label{tab:tabla_arqui_1_CK}
\end{table}

}

\item {\textbf{Conv-Pool-Conv-Pool-FC \underline{\textit{Arquitectura Propuesta}} }

\begin{table}[H]
    \centering
    \includegraphics[width=140mm]{Imagenes/tabla_arqui_2_fer.png} 
    \caption{Evaluación de la arquitectura 2 y sus parámetros, FER2013}
    \label{tab:tabla_arqui_2_fer}
\end{table}

\begin{table}[H]
    \centering
    \includegraphics[width=140mm]{Imagenes/tabla_arqui_2_CK.png}
    \caption{Evaluación de la arquitectura 2 y sus parámetros, CK+}
    \label{tab:tabla_arqui_2_CK}
\end{table}

}

\item {\textbf{Conv-Pool-Pool-Conv-Conv-Pool-FC}

\begin{table}[H]
    \centering
    \includegraphics[width=140mm]{Imagenes/tabla_arqui_3_fer.png} 
    \caption{Evaluación de la arquitectura 3 y sus parámetros, FER2013}
    \label{tab:tabla_arqui_3_fer}
\end{table}

\begin{table}[H]
    \centering
    \includegraphics[width=140mm]{Imagenes/tabla_arqui_3_CK.png}
    \caption{Evaluación de la arquitectura 3 y sus parámetros, CK+}
    \label{tab:tabla_arqui_3_CK}
\end{table}
}
\end{itemize}

De acuerdo con los experimentos realizados, se puede obsevar que la mejor configuracion de capas es: convolucion-\textit{pooling}-convolucion-\textit{pooling}-FC, con 32 y 64 filtros de tamaño 4$\times$4 en la primera y segunda capa de convolucion respectivamente, ventanas de agrupacion de 2$\times$2 en las dos capas de sub-muestreo y un total de 1024$\times$1024 neuronas en las ultimas capas totalmente conectadas.

Segun los resultados mostrados en las tablas~\ref{tab:tabla_arqui_1_fer} y~\ref{tab:tabla_arqui_1_CK} se puede concluir que, la utilización de dos capas consecutivas de convolución es una mala elección, debido a que el hecho de aplicar dos filtros consecutivos hace que la información de la imagen se pierda mas rápidamente que lo normal, ya que el tamaño de la imagen se reduce significativamente después de cada una de estas operaciones. También, las tablas~\ref{tab:tabla_arqui_3_fer} y~\ref{tab:tabla_arqui_3_CK} muestran resultados muy por debajo de la segunda configuración(la mejor obtenida en los experimentos realizados), similar que la explicacion anterior, se concluye que se dan estos resultados debido a la utilizacion de dos capas consecutivas de sub-muestreo, ya que el objetivo de estas capaz es reducir las dimensiones de la imagen solo preservando información reelevante, sin embargo, al realizar una operación de sub-muestreo detras de otra, hace que se pierda mas información de la necesaria. 

\section{Arquitectura Propuesta}
\label{sec:arq_propuesta}
De acuerdo con los experimentados mostrados en la seccion~\ref{sec:experiment}, la arquitectura que consiguio los mejores resultados sigue la siguiente configuracion: la entrada consta de una imagen de 48x48 pixeles en escala de gris, que es el resultado de la fase de detección y extracción de rostro, seguido de una capa de 32 convoluciones con filtro de 4x4 sin solapamiento, luego se aplica un sub muestreo de 2x2 con función MAX \footnote[6]{Función que determina el máximo de n números.} , seguido de una capa de 64 convoluciones con filtros de 2x2 sin solapamiento, para posteriormente aplicar un sub muestreo tambien de dimensiones 2x2 con función MAX, aplicada las convoluciones y sub muestreo se procede a aplicar el Dropout\footnote[7]{Una forma simple de prevenir el \textit{overfitting} en Redes Neuronales..} con 20\%, seguido de dos capas, con 1024 neuronas totalmente conectadas cada una. Finalmente, para la clasificación se aplica la función de normalización Softmax, la cual toma 6 clases, las cuales representar al numero de expresiones a reconocer.
\vspace{0.5cm}
\begin{table}[H]
    \centering
    \includegraphics[width=60mm]{Imagenes/tabla_arquitectura.png}
    \caption{Arquitectura del modelo propuesto}
    \label{tab:tabla_arquitectura}
\end{table}


\begin{figure}[H]
		\centering
		\includegraphics[width=180mm]{Imagenes/arquitectura_CNN_grafico.pdf}
		\caption{Arquitectura grafica del modelo propuesto.}
		\vspace{0.15cm}
		\textit{Fuente: Propio.}
		\label{fig:arquitectura_CNN_grafico}
\end{figure}

La tabla~\ref{tab:tabla_arquitectura} muestra de arriba para abajo, la representación secuencial de las capas dentro de la arquitectura propuesta antes mencionada. Sin embargo, La figura~\ref{fig:arquitectura_CNN_grafico} muestra de una forma mas detallada cada una de las capas que integran la arquitectura, construyendo de esta forma una red neuronal convolucional. Tambien se muestra en esta figura, la salida final obtenida despues de realizar una consulta sobre el modelo, resaltando el rostro detectado en la imagen de entrada, y relacionando la expresion facial que este tiene con una imagen caricaturizada.

\section{Descripción de la Capas de la Arquitectura}

Dentro de cada capa perteneciente a la red neuronal convolucional, se lleva a cabo operaciones de multiplicacion de matrices, agrupacion de regiones, etc, con el objetivo de extraer o resaltar características importantes de las cuales la red pueda aprender patrones y filtros que se adanten mejor al problema, de tal forma que pueda dar respuestas precisas a entradas futuras. 

Siguiendo el orden de capas de la arquitectura propuestas, presentada en la seccion~\ref{sec:arq_propuesta}, se describe el comportamiento de cada uno de ellos de la siguiente forma: 
\begin{itemize}
\item
{
\textbf{Primera capa convolución.} Cuenta con 32 filtros (mapa de características) de tamaño 4x4 pixeles. Esta capa tiene como objetivo extraer características de alto nivel. La figura~\ref{fig:filtro1} muestra el conjunto de imagenes generadas despues de aplicar los 32 filtros de convolución sobre la imagen de entrada, estas imagenes muestra claramente que en esta capa, la arquitectura se encarga de aprender filtros que sean capaces de extraer bordes y partes fundamentales del rostro, tales como: los ojos, boca, nariz y cejas. Esta primera capa de convolución genera una imagen de salida por cada filtro, siendo así, 32 nuevas imágenes las entradas para la siguiente capa. 

\begin{figure}[H]
		\centering
		\includegraphics[width=100mm]{Imagenes/filtro1.png}
		\caption{Imágenes de salida después de la primera convolución.}
		\vspace{0.15cm}
		\textit{Fuente: Propio.}
		\label{fig:filtro1}
\end{figure}
}

\item
{
\textbf{Primera capa de \textit{pooling} o sub-muestreo.} Recibe como parámetros de entrada, las 32 imágenes generadas a partir de la primera capa de convolución. Su función es la de reducir características redundantes mediante la agrupación de pixeles y eleccion del mejor de entre ellos, esta agrupacion se realiza en sub-regiones no solapadas de la matriz, de dimensiones 2$\times$2. Despues de obtener estas sub-regiones, se obtiene el maximo pixel de entre ellas para ser la salida de dicha sub-region. La figura~\ref{fig:filtro2} muestra imágenes semejantes a las obtenidas por la primera capa de convolución, esto se debe a que esta capa no realiza ninguna alteración en el valor de los píxeles, mas al contrario, esta encargada de reducir las dimensiones de las imágenes de entrada, eliminando los píxeles menos significativos. Debido a que se aplica este tipo de operaciones a cada imagen de entrada de esta capa, su número de imágenes de salida sera igual a su numero de entradas.

\begin{figure}[H]
		\centering
		\includegraphics[width=100mm]{Imagenes/filtro2.png}
		\caption{Imágenes de salida después del primer sub-muestreo.}
		\vspace{0.15cm}
		\textit{Fuente: Propio.}
		\label{fig:filtro2}
\end{figure}
}

\item
{
\textbf{Segunda capa de convolución.} Cuenta con 64 filtros y recibe como parámetros de entrada las imágenes generadas a partir de la primera capa de sub-muestreo. A diferencia de la primera capa de convolución, su funcion es la de extraer y detectar caracteristicas de mas bajo nivel. La figura~\ref{fig:filtro3} muestra las 64 imagenes generadas a partir los filtros de esta capa, en ellas se puede observar de manera mas abstracta ciertos puntos y rectas que representan partes del rostro de la imagen de entrada. Como se menciono anteriormente, esta capa origina 64 imágenes de salida, cada una con un tamaño de 21$\times$21 píxeles.

\begin{figure}[H]
		\centering
		\includegraphics[width=100mm]{Imagenes/filtro3.png}
		\caption{Imágenes de salida después de la segunda convolución.}
		\vspace{0.15cm}
		\textit{Fuente: Propio.}
		\label{fig:filtro3}
\end{figure}
}

\item
{
\textbf{Segunda capa de \textit{pooling} o sub-muestreo.} Recibe como parámetros de entrada las imágenes generadas a partir de la segunda capa de convolución. Similar que la primera capa de sub-muestreo, es la encargada de eliminar características no relevantes de las imágenes de entrada, mediante la agrupación y selección del mejor píxel, las sub-regiones de agrupacion son de dimensiones 2$\times$2. La figura~\ref{fig:filtro4} muestra las 64 imágenes generadas a partir de esta capa, donde cada una de estas nuevas imágenes son de tamaño 10x10 pixeles. Segun esta figura se puede observar que el rostro se distorsiona y pierde su forma original, siendo estas, a lo que se conoce como características de bajo nivel.
\begin{figure}[H]
		\centering
		\includegraphics[width=100mm]{Imagenes/filtro4.png}
		\caption{Imágenes de salida después del segundo sub-muestreo.}
		\vspace{0.15cm}
		\textit{Fuente: Propio.}
		\label{fig:filtro4}
\end{figure}
}
\item
{
\textbf{Capas totalmente conectadas.} Recibe como parámetros de entrada las imágenes generadas a partir de la segunda capa de sub-muestreo, las cuales representan las características a bajo nivel pertenecientes a la imagen de entrada original. El objetivo de esta capa, es relacionar cada una de estas características por medio de las 2 capas totalmente conectadas con 1024 neuronas cada una, donde cada neurona es una imagen que representa alguna característica encontrada en las capas anteriores. Al final de la arquitectura, se utiliza la función de normalización \textit{softmax} para generar 6 salidas, donde cada una de ellas representa a una expresión facial. La función de normalización es la encargada de asignar una probabilidad a cada una de estas salidas, siendo nuestra salida final, aquella que presente la mayor probabilidad.
}
\end{itemize}

\section{Parametros de la Arquitectura}

El número de parámetros que contiene una determinada arquitectura, depende mucho del número de capas, filtros, y neuronas en la capa totalmente conectada. Conocer la cantidad de parámetros en una red, es de gran utilidad, debido a que gracias a ellos podemos hacernos una idea de cuan difícil o fácil es entrenar esta. Por tanto, mientras una red o arquitectura tenga mas parámetros, esta sera mas costosa, obviamente, debido a la gran cantidad de parámetros por optimizar. Sin embargo, la cantidad de parámetros no necesariamente refleja la robustés de una red para una determinada tarea. Por ejemplo, puede existir problemas que con pocos parámetros a optimizar sean mejores que considerando complejas arquitecturas, o de forma viceversa.

Una manera mas fácil de entender el comportamiento de la optimización de parámetros, es ver esta red como una función, donde, las salidas o etiquetas de cada entrada son el resultado de la función y cada variables son las entradas, siendo el principal objetivo, optimizar todas las constantes que multiplican a las variables, de tal manera que la función se ajuste para todas las entradas y de las salidas correctas. 

La tabla~\ref{tab:parametros} muestra la cantidad de parámetros obtenidos por la red. Un detalle muy importante, es que la capas de sub-muestreo no contienen ningún tipo de párametro a optimizar, esto es debido a que la función de agrupacion es fija y no necesita ser modificada. Sin embargo, en la capa de convolución los filtros inicialmente son ceros, y son modificados en cada etapa de entrenamiento, lo mismo ocurre con las neuronas de las últimas capas totalmente conectadas. La cantidad total de parametros es 7,620,199, distribuidos de la siguiente forma: 

\begin{itemize}
\item Primera capa con 32 convoluciones: 544 parámetros
\item Segunda capa con 64 convoluciones: 8256 parámetros
\item Primera capa totalmente conectada: 6554624 parámetros
\item Segunda capa totalmente conectada: 1049600 parámetros
\item función de normalización Softmax: 7175 parámetros
\end{itemize}

\begin{table}[H]
    \centering
    \includegraphics[width=160mm]{Imagenes/parametros.png} 
    \caption{Número de parámetros de la arquitectura propuesta.}
    \label{tab:parametros}
\end{table}

	
\section{Entrenamiento de la Red}

Se abordo esta tarea como un problema de aprendizaje supervisado. Como se relato en la sección~\ref{sec:deep-learning}, este tipo de aprendizaje requiere como datos de entrada, imágenes con una respectiva etiqueta, la cual indica la expresión facial a la que representa dicha entrada. Por lo tanto, descrito de una manera muy superficial y general, el proceso de entrenamiento consiste en ingresar imágenes de entradas con sus respectivas etiquetas a la red e intentar que esta aprenda a modelar sus pesos de tal manera que alcance las salidas deseadas. Este proceso se realiza con la ayuda del algoritmo de aprendizaje \textit{backpropagation}, el cual propaga el error final obtenido hacia las neuronas de las capas anteriores que colaboraron en la obtención de dicho resultado.

Como se menciono en el anterior párrafo, la etapa de entrenamiento consiste en ingresar todos los datos y esperar que la red aprenda de ellos por medio de algoritmos de aprendizaje. Sin embargo, generalemente nunca es suficiente una sola iteración de este proceso, por lo que, se repite dicho proceso $x$ veces, siendo $x$ mas conocido como el número de \textit{épocas}. Debido a que este tipo de algoritmos son basados en prueba y error, el entrenamiento se realiza hasta que la red comience a dar señales de sobreajuste de pesos, es decir, el aprendizaje ya no mejora, mas bien empeora. En conclusión, los resultados no necesariamente serán mejores mientras mas sea el número de épocas, como se menciono antes, este tipo de abordage(\textit{deep learning}) en la actualidad no tiene un estudio previo que indique la cantidad de parámetros, capas, épocas y funciones que una red deba tener, por lo tanto, la respuesta esta en la cantidad de experimentos que se realicen, pero debido al poder computacional y tiempo que este requiere, es una limitante para todos los investigadores en el mundo. 

Para los experimentos realizado en la fase de entrenamiento, orientados a resolver el problema planteado(reconocimiento de expresiones faciales), se siguieron los pasos antes descritos y se observó que después de la época 100 la red deja de aprender. Esta descripción se muestra de manera mas detallada (por cada base de datos experimentada) en la sección~\ref{sec:experiment}.

\section{Pruebas al Modelo Creado}

La fase de entrenamiento da origen a un archivo con extensión .h5, el cual almacena los pesos de la red. Ya con este archivo en mano, se realiza la siguiente etapa, que consiste en las pruebas al modelo creado. Estas pruebas o consultas siguen la siguiente secuencia de pasos: 

\begin{itemize}
\item Primero, se lee una imagen como dato de entrada, de dimensiones mayor o igual a 48x48 píxeles. La imagen puede ser a colores o no, el algoritmo internamente lo procesa como una imagen en scala de gris.
\item Segundo, para reducir la cantidad de información irrelevante en la imagen(partes de la imagen que no pertenecen a un rostro: cuerpo, fondo, etc), se utiliza el detector de rostros \textit{Haar Cascade}, obteniendo así, cuatro coordenadas que delimitan el rostro en la imagen de entrada.
\item Tercero, ya con las cuatro coordenadas obtenidas por el algoritmo de detección, se recorta la imagen. Si la imagen recortada resulta ser de tamaño menor que 48$\times$48 píxeles, esta se redimensiona a dicho tamaño. Caso contrario la imagen se mantiene tal cual.
\item Cuarto, la imagen obtenida en el paso anterior es pasada como dato de consulta al modelo. Este devuelve un valor entero en el intervalo $[0-6]$, donde cada uno de esos numeros representa una expresion facial. Finalmente se muestra como imagen de salida una figura caricaturizada que esta relacionada con el número obtenido.
\end{itemize}

La figura~\ref{fig:test} muestra los pasos 3 y 4 mencionadas en los parrafos anteriores.

\begin{figure}[H]
		\centering
		\includegraphics[width=160mm]{Imagenes/test.pdf}
		\caption{Secuencia de pasos en la fase de pruebas.}
		\vspace{0.15cm}
		\textit{Fuente: Propio.}
		\label{fig:test}
\end{figure}

\section{Recopilación de Imágenes de Expresiones Faciales.}
En la actualidada el internet nos ha abastecido con grandes cantidades de informacion. Dentro de estas, podemos encontrar comunidades dedicadas a la investigacion que proveen bases de datos de imagenes etiquetadas relacionadas con problemas especificos, tales como: el reconocimiento de placas de automoviles, clasificacion de imagenes o como en nuestro caso, reconocimiento de expresiones faciales. Estas bases de datos son utilizadas por toda la comunidad cientifica, publicando y comparando sus resultados con trabajos previos y acercandoce mas y mas a un margen de error minimo. Generalemente estas bases de datos cuentan con imagenes cuidadosamente seleccionadas, intentando cubrir todos los casos posibles que se podrian presentar en escenas de la vida real. 

Como se menciono en el parrafo antetior, en este trabajo, la recopilación de las imágenes de expresiones faciales se obtuvo de 2 fuentes secundarias de información (internet) en los cuales los datos están pre-elaborados. La primera base de datos utilizada es contiene imagenes de 48$\times$48 pixeles de tamaño, mientras que la segunda de 640$\times$640. Para la fase de prueba tambien se utilizaron imagenes aleatorias de internet, dentro de estas imagenes se consideran personas del mundo real y caricaturizadas(ver figura~\ref{fig:caricaturizado}).

\begin{figure}[H]
		\centering
		\includegraphics[width=110mm]{Imagenes/pruebas.pdf}
		\caption{Ejemplo de un rostro real y caricaturizado.}
		\vspace{0.15cm}
		\textit{Fuente: Internet.}
		\label{fig:caricaturizado}
\end{figure}

\section{Base de Datos}
Para la realizacion de experimentos se utilizaron 3 bases de datos, dos de ellas se encuentran disponibilizadas en internet(\textit{FER103} y \textit{CK+}) y una tercera que es resultado de la union de estas dos anteriores. Se opto por considerar esta tercera base de datos, debido a que el abordage de \textit{deep learning} trabaja mejor mientras tenga mas datos para la fase de entrenamiento.
 
\subsection{FER2013}
Es una base de datos del sitio web \textit{Kaggle}, destinada para el concurso de reconocimiento de expresiones faciales. Fue preparada por Pierre-Luc Carrier y Aaron Courville, como parte de un proyecto de investigacion en curso. Esta clasificada en 7 categorias: molesto, disgustado, miedo, feliz, triste, sorpresa y neutro, sin embargo, en este trabajo se opto por unir la categoria enojado y disgustado debido a las similitudes que ambas tieenen.

Este conjunto de datos esta disponible como archivos \textit{train.csv} y \textit{test.csv}, los cuales contienen 2 columnas: \textit{emocion} y \textit{pixels}. La columna \textit{emocion} contiene un codigo numerico en el rango de 0 a 6 inclusive, para la expresion que esta presente en la imagen. La columna \textit{pixels} contiene un string de numeros que representa los valores para cada pixel dentro de la imagen. El conjunto de entrenamiento consiste de 28,709 ejemplos, mientras que los datos de prueba son 3589. La figura~\ref{fig:imagenes_fer} muestra ejemplos de imagenes de este conjunto de datos.


\begin{figure}[H]
		\centering
		\includegraphics[width=110mm]{Imagenes/imagenes_fer.pdf}
		\caption{Imágenes de la base de datos FER2013}
		\vspace{0.15cm}
		\textit{Fuente: Kaggle.}
		\label{fig:imagenes_fer}
\end{figure}

\section{CK$+$}
La base de datos CK+ (Cohn-Kanade) posee imágenes de expresiones faciales frontales de 210 diferentes personas. Estas imágenes tienen una resolución original de 640x490 y 640x480 píxeles. Debido a que solo se encontraron imágenes pertenecientes a esta base de datos y sin ninguna etiqueta que relacione la expresión facial que representa cada una de ellas, en este trabajo se optó por realizar la etiquetación de forma manual(basandonos en el trabajo de Paul Ekman), obteniendo así 3289 imágenes, transformandolas a imágenes en escala de gris con dimensiones de 48x48 píxeles. Similar que la anterior base de datos, se consideraron 6 categorías (enojado, miedo, feliz, triste, sorprendido y neutro). Se utilizó el 80\% del total de imagenes para la fase de entrenamiento (2966 imágenes) y el 20\% para la fase de pruebas (323 imágenes). La figura~\ref{fig:imagenes_ck+} muestra ejemplos de imagenes d eeste conjunto de datos.

\begin{figure}[H]
		\centering
		\includegraphics[width=110mm]{Imagenes/imagenes_ck+.pdf}
		\caption{Imágenes de la base de datos CK+}
		\vspace{0.15cm}
		\textit{Fuente: Base de datos CK+.}
		\label{fig:imagenes_ck+}
\end{figure}


\section{FER2013 - CK$+$}
Esta base de datos resulta de la unión de las dos anteriores (Fer2013 y CK+), obteniendo así un total de 39176 imágenes en escala de gris de dimensiones de 48x48 píxeles. El conjunto de entrenamiento es el 80\% del total (35264 imagenes), mientra que el conjunto de prueba es el 20\% (3912 imagenes). Se realizo esta unión de datos con la finalidad de obtener un modelo mas robusto, ya que las tecnicas de \textit{deep learning} dependen mucho de la cantidad de datos para el entrenamiento de un modelo.

\section{Resultados Experimentales}
\label{sec:experiment}

Los experimentos se realizaron sobre los tres conjuntos de datos mencionados en la seccion anterior. En esta seccion se muestran los resultados de dichos experimentos uno por uno, donde se podran apreciar los niveles de precision alcanzados por cada categoria - expresion facial, asi como sus matrices de confusion, las cuales nos ayudaran para un mejor entendimiento e interpretacion de los resultados. Tambien son mostrados por medio de grafico, el aprendizade de la red medido en terminos de precision durante cada epoca.

Para una mejor comprensión e interpretación de los resultados, se definen las siguiente palabras: \textit{precisión} es la fracción de instancias(datos de entrada) que alcanzaron un resultado correcto entre el número de instancias totales, \textit{recall} es la cantidad de intancias correctas recuperadas sobre la cantidad total de instancias relevantes(falsos positivos y falsos negativos). La figura~\ref{fig:precisionRecall} muestra de una manera mas sencilla ambas definiciones. Por otro lado \textit{f1-score} es la media armónica de la precisión y el \textit{recall}, la ecuación~\ref{eq:f1-score} muestra su representación matemática. \textit{Falsos positivos} o \textit{falses positives} en ingles, son el conjunto de instancias cuyas respuestas se creen que son correctas cuando en realidad estan erradas. \textit{Falsos negativos} o \textit{false negatives} a diferencia del anterior, son el conjunto de instancias que creen que estan errados cuando en realidad son correctos. \textit{verdaderos positivos} y \textit{verdaderos negativos} son simplemente instancias que son o bien correctas o erroneas.

\begin{figure}[H]
		\centering
		\includegraphics[width=120mm]{Imagenes/Precisionrecall.pdf}
		\caption{Precision y recall.}
		\vspace{0.15cm}
		\textit{Fuente: Precision and recall, Wikipedia.}
		\label{fig:precisionRecall}
\end{figure}

\begin{equation}\label{eq:f1-score}
F1 = 2\times\frac{precision\times recall}{precision + recall}
\end{equation}

\subsection{FER2013}

\begin{table}[H]
    \centering
    \includegraphics[width=100mm]{Imagenes/tabla_resultados_fer.png} 
    \caption{Resultados obtenidos - FER2013}
    \label{tab:tabla_resultados_fer}
\end{table}

En la tabla~\ref{tab:tabla_resultados_fer} se muestra los resultados obtenidos durante los experimentos realizados en la base de datos FER2013. Dentro de los niveles de precision y \textit{recall}, se observa que las categorias 'feliz' y 'sorpresa' obtienen mejores resultados que las demas, alcanzando un nivel de precision de 74\% y 71\% respectivamente. Por otro lado, la categoria 'enojado' alcanza 46\%, siendo la mas baja de entre todas. El resultado general alcanzado en este conjunto de datos es de 57\%.

\begin{figure}[H]
		\centering
		\includegraphics[width=90mm]{Imagenes/matriz_confusion_fer.png}
		\caption{Matriz de confusión, precisión del Test - FER2013}
		\label{fig:matriz_confusion_fer}
\end{figure}

Gracias a la matriz de confusión(figura~\ref{fig:matriz_confusion_fer}) se puede descubrir con exactitud la cantidad de datos que fallan y aciertan por categoría. Como se menciono anteriormente, la categoria 'feliz' es la que presenta mayor precisión, acertando 659 imágenes de las 879 en total, también podemos observar que la red confunde esta categoría en mayor cantidad con la categoría 'neutro', siendo en total 68 imagenes de expresiones faciales 'neutras' confundidas con expresiones de la categoría 'feliz'. Otro detalle muy importante que nos muestra esta imagen, es que la red confunde en grandes cantidades las categorías 'triste' con 'enojado' y 'neutro', lo que nos da a entender que deberiamos poner mas enfasis e importancia en imagenes pertenecientes a estas categorias.  
\begin{figure}[H]
		\centering
		\includegraphics[width=95mm]{Imagenes/precision_fer.pdf}
		\caption{Precisión durante el proceso de entrenamiento y prueba (\%) – FER2013}
		\label{fig:precision_fer}
\end{figure}

La figura~\ref{fig:precision_fer} muestra la precisión alcanzada por la red durante el proceso de entrenamiento y pruebas en el transcurso de cada época. La linea azul muestra el crecimiento del aprendizaje con los datos de entrenamiento, se puede observar que entre la época 50 y 100 llega a alcanzar el 100\% de precisión, sin embargo esto sucede por que se realiza este cálculo con datos que ya fueron visualizados por la red. La linea verde muestra la precisión alcanzada con los datos de prueba, alcanzando un 57\%, esto sucede por que los datos de prueba son datos nuevos para la red, datos que nunca fueron vistos durante su etapa de entrenamiento.

 
\subsection{CK+}

\begin{table}[H]
    \centering
    \includegraphics[width=100mm]{Imagenes/tabla_resultados_ck+.png} 
    \caption{Resultados obtenidos - CK+}
    \label{tab:tabla_resultados_ck+}
\end{table}

La tabla~\ref{tab:tabla_resultados_ck+} muestra los resultados obtenidos durante los experimentos realizados en la base de datos CK+. Dentro de los niveles de precision y \textit{recall}, se observa que las categorias 'feliz' y 'sorpresa' obtienen mejores resultados que las demas(de igual forma que los resultados en la anterior base de datos), alcanzando un nivel de precision de 100\% y 100\% respectivamente. Por otro lado, la categoria 'miedo' alcanza 46\%, siendo la mas baja de entre todas. El resultado general alcanzado en este conjunto de datos es de 91\%.

\begin{figure}[H]
		\centering
		\includegraphics[width=90mm]{Imagenes/matriz_confusion_ck+.png}
		\caption{Matriz de confusión, precisión del Test - CK+}
		\label{fig:matriz_confusion_ck+}
\end{figure}

La figura~\ref{fig:matriz_confusion_ck+} muestra la matriz de confusión correspondiente a los resultados obtenidos para esta base de datos. Como se mostró en la tabla anterior, se puede observar que de las 82 muestras de la categoría 'feliz', todas son acertadas, alcanzando así el 100\% de precisión mencionado anteriormente. También se puede observar que en la categoría 'miedo' de las 88 imágenes en total, confunde 8 de ellas con la categoría 'enojado'. Otra de las categorías que también es confundida con las demas, es 'neutro', donde de 46 imágenes en total, 5 son confundidas con 'sorpresa', 4 con 'enojo' y 1 con 'feliz'.
 
\begin{figure}[H]
		\centering
		\includegraphics[width=95mm]{Imagenes/precision_ck+.pdf}
		\caption{Precisión durante el proceso de entrenamiento y prueba (\%) - CK+}
		\label{fig:precision-ck+}
\end{figure}

La figura~\ref{fig:precision-ck+} muestra la precisión alcanzada por la red durante el proceso de entrenamiento y pruebas en el transcurso de cada época. La linea azul representa la precisión alcanzada con los datos de entrenamiento, donde se observar claramente que este alcanza el 100\% de precisión aproximadamente en la época 50, sim embargo, también esta curva presenta oscilaciones en ciertas épocas, bajando ligeramente su nivel de precisión. La linea verde representa la precisión alcanzada con el conjuto de pruebas, el cual muestra el pico mas alto de la curva entre la época 200 y 300, siendo este número de épocas el apropiado para obtener la mejor precisión posible.

\subsection{FER2013 - CK+}

\begin{table}[H]
    \centering
    \includegraphics[width=100mm]{Imagenes/tabla_resultados_fer_ck+.png} 
    \caption{Resultados obtenidos - FER2013 - CK+}
    \label{tab:tabla_resultados_fer_ck+}
\end{table}

La tabla~\ref{tab:tabla_resultados_fer_ck+} muestra los resultados obtenidos durante los experimentos realizados en la union de las bases de datos FER2013 y CK+. Dentro de los niveles de precision y \textit{recall}, se observa que las categorias 'feliz' y 'sorpresa' obtienen mejores resultados que las demas(de igual forma que en las anteriores bases de datos), alcanzando un nivel de precision de 75\% y 77\% respectivamente. Por otro lado, la categoria 'triste' alcanza 44\%, siendo la mas baja de entre todas. El resultado general alcanzado en este conjunto de datos es de 60\%.

\begin{figure}[H]
		\centering
		\includegraphics[width=90mm]{Imagenes/matriz_confusion_fer_ck+.png}
		\caption{Matriz de confusión, precisión del Test FER2013 - CK+}
		\label{fig:matriz_confusion_fer_ck+}
\end{figure}

La figura~\ref{fig:matriz_confusion_fer_ck+} muestra la matriz de confusion correspondiente a la union de las dos bases de datos antes mencionadas. En ella se puede observar que muchas categorias son confundidas por las otras en cantidades considerables, sin embargo la categoria 'triste' es la que mas errores comete, siendo 109 y 106 imagenes confundidas con la categorias 'miedo' y 'neutro' respectivamente. 
  
\begin{figure}[H]
		\centering
		\includegraphics[width=95mm]{Imagenes/precision_fer_ck+.pdf}
		\caption{Precisión durante el proceso de entrenamiento y prueba (\%) FER2013 - CK+}
		\label{fig:precision_fer_ck+}
\end{figure}

En la figura~\ref{fig:precision_fer_ck+}, similar que en las dos anteriores gráficas de precisión correspondientres a las anteriores bases de datos, se puede observar que con los datos de entrenamiento se obtiene un nivel de precisión del 100\$, sin embargo, con los datos de entrenamiento se genera una curva con pequeñas oscilaciones durante el transcurso de las épocas, por lo tanto se concluye que utilizando un número de épocas mayor o igual a 20 se obtendra el máximo nivel de precisión.






\chapter*{Resultados Generales}

Se obtuvo un nivel de precisión de 91\% en los experimentos realizados con la base de datos CK+ (sub-seccion~\ref{subsec:ck+}), 57\% con FER2013 (sub-seccion~\ref{subsec:fer2013}) y 60\% con la unión de ambas bases de datos (sub-seccion~\ref{subsec:ck+fer2013}), estando 14\% debajo del nivel de precisión del primer lugar del concurso mundial de reconocimiento de expresiones faciales organizado por \textit{Kaggle} (utilizando la base de datos FER2013). Se presume que dichos resultados no alcanzaron ni superaron métodos del estado del arte, debido a la cantidad de experimentos realizados por las limitaciones de Hardware que se presentaron en el desarrollo de este trabajo (uso de CPU más no de GPU en la fase de entrenamiento). Sin embargo, los resultados obtenidos con la base de datos CK+ son prometedores, debido a la variedad de imágenes de expresiones faciales  que este conjunto de datos posee.

\renewcommand\bibname{Resultados}
\addcontentsline{toc}{chapter}{Resultados}
\chapter*{CONCLUSIONES}
\begin{itemize}
\item El desarrollo de una arquitectura de Red Neuronal Convolucional es muy
compleja, debido a que no se cuenta con fundamentos teóricos para la correcta
selección de parámetros (filtro de convolución, submuestreo, neuronas, etc.). La
arquitectura propuesta muestra resultados con niveles de precisión alto, lo cual da
evidencia que se hizo una correcta selección de las capas y los parámetros que lo
componen.
\item Existen dos formas de recopilación de datos: La primera consiste en crear una
propia base de datos lo cual requiere de tiempo y dinero, la segunda opción y por
la que se optó en este proyecto, consiste en extraer datos de internet de
organizaciones dedicadas al campo de estudio.

\item El uso de 2 capas de convolución con 64 y 32 filtros de tamaños 4x4 y 2x2 pixeles
respectivamente muestra que es una buena selección de parámetros para la
extracción de características de expresiones faciales.

\item El Submuestreo o Pooling cumple funciones importantes relacionadas con el coste
computacional, reduciendo el número de operaciones de computo con la
disminución de las dimensiones de la imagen con el fin de reducir características.
La utilización de 2 capas de Submuestreo de tamaño de agrupación 2x2 pixeles y
con función Max, muestra que es una buena selección de parámetros para la
reducción de características.

\item La función de activación RELU resulta ser la mejor opción para la
implementación de una arquitectura de Red Neuronal por los resultados mostrados
en el estado del arte del Deep Learning.

\item La función de normalización softmax es muy eficiente para la clasificación de
múltiples clases por los resultados mostrados en el estado del arte del Deep
Learning.

\item Se entrenó satisfactoriamente la Red Neuronal Convolucional (basándonos en la
técnica early stopping - Ver anexos), construida a partir de las capas antes
mencionadas, teniendo algunas limitaciones, sea el caso de recursos
computacionales, ocasionando demoras para la fase de entrenamiento, ya que solo
se contó con el uso de CPU mas no de GPU.

\item En la base de datos de datos CK+ se obtuvo un nivel de precisión alto por que las
imágenes muestran rasgos resaltantes de las expresiones faciales los cuales fueron
etiquetados manualmente en esta investigación. En FER2013 se muestra un nivel
de precisión no muy bueno (Tabla 2) por el desbalance de datos en algunas
categorías y para nivelarlos y alcanzar mejores resultados (Tabla 4), una opción
es combinar con otras bases de datos, pero se corre riesgo de que los criterios de
etiquetado de las expresiones faciales difieran.


\end{itemize}
\renewcommand\bibname{Conclusiones}
\addcontentsline{toc}{chapter}{Conclusiones}
\chapter*{RECOMENDACIONES}

Por la experiencia adquirida en la realización de este proyecto de investigación, se lista un conjunto de sugerencias útiles para los lectores.

\begin{itemize}


\item Se recomienda obtener un equipamiento de \textit{hardware} adecuado para trabajar con \textit{deep learning}. Principalmente la adquisición de \textit{GPUs} que facilitarán el entrenamiento de redes neuronales profundas por su poder de paralelización en operaciones con matrices. Caso no exista la posibilidad de adquirir un equipamiento propio para \textit{deep learning}, es recomendado usar una computadora con capacidad minima de 8GB  de RAM. Sin embargo, su capacidad estará limitada a redes neuronales con menor profundidad. Existe una opción adicional,  alquilar servidores en la nube especializados en el entrenamiento de arquitecturas de \textit{deep learning}. La desventaja de estos servicios es que son de pago.

\item En el proceso de investigación, se recomienda que la información extraída sea de fuentes confiables. Para ello, sugerimos que accedan a material de investigación (papers. artículos científicos y otros) de instituciones prestigiosas como la IEEE, ACM, SPRINGER y otros.

\item Respecto a las herramientas para implementación,  recomendamos usar Python como lenguaje de programación por las facilidades que brinda y por su uso concurrido a nivel mundial. Además, de tener muchas bibliotecas que facilitan el trabajo con \textit{deep learning}.

\end{itemize}
\renewcommand\bibname{Recomendaciones}
\addcontentsline{toc}{chapter}{Recomendaciones}
\chapter*{TRABAJOS FUTUROS}

A futuro se tiene pendiente el reconocimiento de expresiones faciales en tiempo
real, reemplazando el detector de rostros Haar Casacade por uno basado en Deep
Learning y para nivelar el desbalance de datos por categoría se tiene pensado utilizar data
augmentation.

\renewcommand\bibname{TrabajosFuturos}
\addcontentsline{toc}{chapter}{Trabajos Futuros}

\end{spacing}


%----------------------bibliografia

%\cleardoublepage
\addcontentsline{toc}{chapter}{Bibliografía}
\begin{spacing}{1.0}
\nocite{*}
\bibliographystyle{abbrv} % estilo de la bibliografía.
\bibliography{bibli.bib} % yyyy.bib es el fichero donde está salvada la bibliografía.

\end{spacing}


%----------------------------------apendix

\appendix

\part*{ANEXOS}
\addcontentsline{toc}{chapter}{ANEXOS}

\chapter{OTROS CONCEPTOS}\label{A}
\section{Matriz de Confusión}

La matriz de confusion es una técnica que facilita el análisis del rendimiento de un algoritmo de clasificación. Frecuentemente, el rendimiento de un clasificador es solo analizado en términos de la \textit{precision} (verdaderos positivos, verdaderos negativos y falsos positivos). Sin embargo, este no refleja completamente el verdadero rendimiento del clasificador. Así, comunmente es considerado el \textit{recall} para solucionar parcialmente este problema. El \textit{recall} toma en cuenta los falsos negativos (repuestas predichas como falsas cuando en realidad son verdaderas). Adicionalmente, para evaluar el rendimiento final de un clasificador en forma completa se usa el \textit{f1-score}. El f1-score toma en consideración tanto la \textit{precision} y el \textit{recall}, obtiendo una penalización del rendimiento del clasificador cada vez que este se equivoque \cite{30Mconfusion}.
Así, la matriz de confusión cumple un papel importante en el calculo de \textit{precision}, \textit{recall} y \textit{f1-score}. En la Tabla \ref{tab:estruc_confusion_matrix} se puede observar la estructura general de una  matriz de confusión. así mismo en las ecuaciones \ref{eq:Aprecision} y \ref{eq:Arecall}  se muestra el uso de los elementos la matriz de confusión para calcular las métricas de \textit{precision} y \textit{recall} respectivamente. Finalmente, la ecuación \ref{eq:Af1score} muestra el cálculo de \textit{f1-score}. 


\begin{table}[!htb]
  
  \noindent
  \renewcommand\arraystretch{1.5}
  \setlength\tabcolsep{0pt}
  \begin{center}
  \begin{tabular}{c >{\bfseries}r @{\hspace{0.7em}}c @{\hspace{0.4em}}c @{\hspace{0.7em}}l}
    \multirow{10}{*}{\rotatebox{90}{\parbox{1.1cm}  { \bfseries\centering  Valores \\ predichos}}} & 
      & \multicolumn{2}{c}{\bfseries Valores referenciales} & \\
    & & \bfseries Positivos & \bfseries Negativos \\
    & {Positivos} & \MyBox{True}{Positive (TP)} & \MyBox{False}{Negative (FN)} \\[2.4em]
    & Negativos & \MyBox{False}{Positive (FP)} & \MyBox{True}{Negative (TN)} \\
  
  \end{tabular}
  \end{center}
    \caption{Estructura general de la matriz de confusión}
        \label{tab:estruc_confusion_matrix}
\end{table}

\begin{equation}\label{eq:Aprecision}
precision = \frac{ TP}{TP+FP}
\end{equation}

\begin{equation}\label{eq:Arecall}
recall = \frac{ TP}{TP+FN}
\end{equation}

\begin{equation}\label{eq:Af1score}
f1-score = \frac{2 \cdot precision\cdot recall}{precision+ recall}
\end{equation}



\section{Machine Learning}

El \textit{machine learnin}g es un area de la la inteligencia artificial que alberga algoritmos de aprendizaje. La definición: “Se dice que un programa de computadora aprende de la experiencia E con respecto a alguna clase de tareas T y la medida de rendimiento P, si su desempeño de tareas en T, medido por P, mejora con la experiencia E.” realizada en \cite{mitchell1997machine} nos explica que significa aprender para una máquina.
 
\section{Aplicaciones de Machine Learning}

Los algoritmos de \textit{machine learning} 
pueden ser usados para una infinidad de aplicaciones que solucionen problemas del mundo real. La amplia gama de aplicaciones en los que pueden ser utilizados los algoritmos de \textit{machine learning} van desde motores de búsqueda, diagnósticos médicos, reconocimiento del habla y del lenguaje, robótica, etc.
Se mencionan algunas de las aplicaciones con mayor tendencia que tienen como elemento principal el uso de algoritmos de \textit{machine learning} \cite{31MLApplications}:


 


\begin{itemize}
\item Detección de rostro.
\item Reconocimiento facial, de voz o de objetos.
\item Motores de Busqueda.
\item Anti-spam.
\item Anti-virus.
\item Genética.
\item Predicción y pronósticos del clima.
\item Comprensión de textos.
\item Vehículos autónomos y robots.
\item Análisis de imágenes de alta calidad.
\item Análisis de datos económicos.
\item Análisis de comportamiento de consumo y productividad.

\end{itemize}


\section{Early Stopping}

%https://deeplearning4j.org/earlystopping

%http://page.mi.fu-berlin.de/prechelt/Biblio/stop_tricks1997.pdf

%https://arxiv.org/pdf/1703.09580.pdf


Cuando se entrena una red neuronal se tiene que definir a priori cuantas iteraciones se harán sobre el conjunto de datos de entrenamiento. 
Depende fuertemente del número de épocas escogido si se obtendrá un buen ajuste a los datos o por el contrario podra ocurrir sobreajuste en los datos de entrenamiento. 

El método de \textit{early stopping}  ayuda a resolver el problema de escoger el número de épocas en el que logra un buen ajuste a los datos de entrenamiento \cite{32Stopping}, \cite{prechelt1998early}.

Los pasos a seguir para la aplicación del método de early stopping son los siguientes:

\begin{itemize}
 \item Dividir la base de datos en 2 partes: base de datos de entrenamiento y base de datos de validación.
 
 \item Para cada N épocas debemos hacer lo siguiente:
   \begin{itemize}
    \item Evaluar el error del modelo en el conjunto de validación.
    \item Parar el entrenamiento si el error de validación es mayor respecto al anterior error de validación.
    \item Finalmente, se selecciona el modelo creado antes del aumento del error de validación.
   \end{itemize}
\end{itemize}

En la Figura \ref{fig:early_stopping} se puede observar la relación época y rendimiento en el conjunto de entrenamiento y el conjunto de validación. Así, como el punto óptimo en el que se alcanza el mejor rendimiento en el conjunto de validación.


\begin{figure}[!htb]
    \centering
    \includegraphics[width=100mm]{Imagenes/early_stopping.png}
    \caption{Representación gráfica de Early Stopping \\ \textit{Fuente: deeplearning4j \cite{32Stopping}}}
    \label{fig:early_stopping}
\end{figure}




\chapter{TESTING}
\section{Pruebas satisfactorias}
\begin{frame}

\begin{figure}[!htbp]
    
    \begin{multicols}{2}
     \includegraphics[angle=0,width=48mm]{Imagenes/test1.png}
       \caption{Test 1}
       \label{fig:test1}   
       
       \includegraphics[angle=0,width=55mm]{Imagenes/test2.png}
           \caption{Test 2}
           \label{fig:test2} 
           
    \end{multicols}
        
\end{figure}
\end{frame}


\begin{frame}

\begin{figure}[!htbp]
    
    \begin{multicols}{2}
     \includegraphics[angle=0,width=50mm]{Imagenes/test3.png}
       \caption{Test 3}
       \label{fig:test3}   
       
       \includegraphics[angle=0,width=50mm]{Imagenes/test4.png}
           \caption{Test 4}
           \label{fig:test4} 

           
            \includegraphics[angle=0,width=50mm]{Imagenes/test5.png}
              \caption{Test 5}
              \label{fig:test5}   
              
              \includegraphics[angle=0,width=50mm]{Imagenes/test6.png}
                  \caption{Test 6}
                  \label{fig:test6}   
                  
                  
    \end{multicols}
        
\end{figure}
\end{frame}


\begin{frame}

\begin{figure}[!htbp]

\begin{multicols}{2}
   
\includegraphics[angle=0,width=64mm]{Imagenes/test7.png}
    \caption{Test 7}
    \label{fig:test7} 
 
 \includegraphics[angle=0,width=50mm]{Imagenes/test8.png}
     \caption{Test 8}
     \label{fig:test8} 
 \end{multicols}       
\end{figure}

\end{frame}



\section{Pruebas Fallidas}
\begin{frame}


\begin{figure}[!htbp]
    
    \begin{multicols}{2}
     \includegraphics[angle=0,width=42mm]{Imagenes/test9.png}
       \caption{Test 9}
       \label{fig:test9}   
       
       \includegraphics[angle=0,width=42mm]{Imagenes/test10.png}
           \caption{Test 10}
          
           \label{fig:test10} 
           
        \includegraphics[angle=0,width=46mm]{Imagenes/test11.png}
                   \caption{Test 11}
                   \label{fig:test11} 
                   
     \includegraphics[angle=0,width=46mm]{Imagenes/test12.png}
                \caption{Test 12}
                \label{fig:test12}    
    \end{multicols}
        
\end{figure}
\end{frame}




\chapter{HERRAMIENTAS}
Las herramientas de software y hardware son:

\section{Software}
\begin{itemize}
 \item Sistema Operativo. Linux 14.0
 \item Lenguaje de programacion: Python 2.7
 \item Framework Deep Learning: Keras
\end{itemize}

\section{Hardware}
\begin{itemize}
 \item CPU: Intel XEON 3.4 GHz
 \item RAM: 8GB
\end{itemize}

\chapter{GLOSARIO}
\begin{itemize}
\item \textbf{Convolución:} Operador matemático que transforma 2 funciones en una tercera función.
\item \textbf{Modelo:} Representación abstracta, conceptual, gráfica, física o matemática, de
fenomenos, sistemas o procesos a fin de analizarlos, describirlos, explicarlos, simularlos
y predecirlos.
\item \textbf{Arquitectura}: Técnica y estilo con lo que se diseña, proyecta y construye un modelo.
\item 
\textbf{Gradiente:} Derivada parcial de una funcion respecto a cada variable de está.

\item \textbf{Exabyte:} Unidad de medida de almacenamiento de datos cuyo símbolo es el 'EB' equivalente a $10^18$ bytes.

\item \textbf{Kaggle:} Plataforma online que ofrece a sus usuarios la opción de participar en distintas
competencias cuyo principal tema es el análisis de gran cantidad de datos.


\item \textbf{DL:} Deep Learning, es un conjunto de algoritmos en aprendizaje automático que intenta
modelar abstracciones de alto nivel en datos usando arquitecturas compuestas de
transformaciones no-lineales múltiples.



\item \textbf{CNN:} Convolutional Neural Network, tipo de red neuronal artificial donde las neuronas
corresponden a campos receptivos de una manera muy similar a las neuronas en la corteza
visual primaria de un cerebro biológico.

\item \textbf{ML:} Machine Learning, es una rama de la inteligencia artificial cuyo objetivo es
desarrollar técnicas que permitan a las computadoras aprender.


\item \textbf{RNA:} Red Neuronal Artificial, modelos matemáticos, computacionales, artificiales,
ideales de una red neuronal empleados en estadística, psicología cognitiva, e inteligencia
artificial.


\item \textbf{GPU:} Graphics Processor Unit, es un coprocesador dedicado al procesamiento de
gráficos u operaciones de coma flotante, para aligerar la carga de trabajo del procesador
central.

\item \textbf{ImageNet:} Base de datos de imágenes a gran escala con 1.2M de imágenes.


\end{itemize}

\chapter{ACRÓNIMOS}
\begin{itemize}


\item \textbf{ACM:} Association for Computing Machinery, es una sociedad científica y educativa en
ciencias de la computación.

\item \textbf{Mm:} Milímetro, medida de longitud que es igual a la milésima parte de un metro.

\item \textbf{IEEE:} Institute of Electrical and Electronics Engineers, es la mayor organización
profesional técnica del mundo dedicada al avance de la tecnología en beneficio de la
humanidad

\item \textbf{CVPR:} Conference on Computer Vision and Pattern Recognition.

\item \textbf{VLSI:} Sigla en inglés de Very Large Scale Integration

\end{itemize}

%-------------------------------

\end{document}
