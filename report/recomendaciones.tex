\chapter*{RECOMENDACIONES}

Por la experiencia adquirida en la realización de este proyecto de investigación, se lista un conjunto de sugerencias útiles para los lectores.

\begin{itemize}


\item Se recomienda obtener un equipamiento de hardware adecuado para trabajar con \textit{deep learning}. Principalmente la adquisición de GPUs que facilitaran el entrenamiento de redes neuronales profundas por su poder de paralelización en operaciones con matrices. Caso no exista la posibilidad de adquirir un equipamiento propio para \textit{deep learning}, es recomendado usar una computadora con capacidad minima de 8GB  de RAM. Sin embargo, su capacidad estara limitada a redes neuronales con menor profundidad.

\item En el proceso de investigación, se recomienda que la información extraida sea de fuentes confiables. Para ello, sugerimos que accedan a material de investigación (papers. artículos científicos y otros) de instituciones prestigiosas como la IEEE, ACM, SPRINGER y otros.

\item Respecto a las herramientas para implementación,  recomendamos usar Python como lenguaje de programación por las facilidades que brinda y por su uso concurrido a nivel mundial. Además, de tener muchas bibliotecas que facilitan el trabajo con \textit{deep learning}. También tiene una buena documentación lo que facilita su uso. 

\end{itemize}