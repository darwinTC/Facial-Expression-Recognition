\chapter{Aspectos Generales}
\section{Aspectos Generales}
\subsection{Descripción del Problema}
Uno de los sentidos más importantes de los seres humanos es la visión. Ésta es empleada para obtener la información visual del entorno físico. Según Aristóteles, “Visión es saber que hay y donde mediante la vista”. De hecho, se calcula que más del 75\% de las tareas del cerebro son empleadas en el análisis de la información visual. El refrán popular de “Una imagen vale más que mil palabras” tiene mucho que ver con los aspectos cognitivos de la especie humana. Casi todas las disciplinas científicas emplean utillajes gráficos para transmitir conocimiento\footnote[1]{Visión Humana, fuente: Sistemas Adaptativos y Bioinspirados en Inteligencia Artificial\href{http://sabia.tic.udc.es/}{(S.A.B.I.A.)}}.

Uno de los más grandes concursos a nivel mundial en clasificación de imágenes reporta que las técnicas tradicionales (como las técnicas de extracción de características en imágenes estáticas: \textit{Principal component analysis} (PCA), \textit{Edges detector}, \textit{Gabor waveled}; Video: PCA, \textit{Discrete cosine transform} (DCT), \textit{optical flow} e \textit{image difference}) están siendo superadas por técnicas de \textit{Deep Learning} que son basadas en el proceso cerebral humano. Dicho éxito se debe a que las técnicas tradicionales requieren de un ambiente controlado y no son tolerables a cambios como: traslación, rotación y escalado. Por otro lado las técnicas de \textit{Deep Learning} demuestran ser mas robustas y efectivas frente a estos tipos de cambios\footnote[2]{\textit{Deep Learning} vs. \textit{Machine Learning} fuente: \href{http://www.image-net.org/}{Analytics Vidhya}}.
	
Diversas actividades cotidianas necesitan del reconocimiento de imágenes, tal es el caso del reconocimiento de expresiones faciales, que en los últimos años se ha convertido en una de las tareas más estudiadas por investigadores en todo el mundo, con el fin de alcanzar un margen de error minimo para posteriormente centrarse en el desarrollo de aplicaciones en distintos campos como: estudio de marketing, interacción hombre-computador, psicología y análisis educativo \footnote[3]{Las expresiones faciales de las emociones, historia y aplicaciones, fuente: \href{http://medina-psicologia.ugr.es/cienciacognitiva/?p=664}{Ciencia Cognitiva}}, las cuales han sido abordada por diferentes técnicas tradicionales no obteniendo resultados prometedores en imágenes reales que contienen distintos tipos de variaciones (mencionados en el párrafo anterior). Limitando así, la implementación y desarrollo de aplicaciones útiles para el bien común (aplicaciones antes mencionadas).

\subsection{Identificación del Problema}
Las técnicas tradicionales para el reconocimiento de expresiones faciales usadas en la actualidad necesitan de un ambiente controlado (iluminación constante, alta calidad de imagen, poco ruido, imagen sin oclusión), y no son tolerables a cambios como rotación, traslación, escalado. Limitando así, la creación de aplicaciones con imágenes del mundo real (imágenes obtenidas a partir de cámaras de seguridad). Por lo que hay la necesidad de usar nuevas técnicas del estado del arte que nos permitan obtener mejores resultados superando así la limitación antes mencionada.

\section{Antecedentes}
Se muestra una lista de trabajos resaltantes que hacen uso de técnicas de \textit{deep learning}, los cuales sirvieron de inspiración y fuente de información valiosa en el desarrollo de este trabajo. También se presentan trabajos dedicados al reconocimiento de expresiones faciales utilizando técnicas tradicionales de visión por computador y \textit{machine learning}.

\subsection{Técnicas Tradicionales para el reconocimiento de expresiones faciales}
Muchos abordages fueron propuestos para la tarea de reconocimiento de expresiones faciales basados técnicas tradicionales y \textit{machine learning}. 

\begin{enumerate}
\item {\textbf{“Learning active facial patches for expression analysis” \cite{zhong2012learning}}

Zhong et al. propone un método para el reconocimiento de expresiones faciales basado en \textit{patches} informativos de expresiones faciales (e.g. ojos, mejillas, boca, etc) de los rostros humanos contenidos en las imágenes. Así, introduce un marco de aprendizaje por tareas multiples (MTSL) de dos etapas para ubicar de manera eficiente los \textit{patches} discriminativos en las imagenes con contenido facial. La primera etapa consiste en extraer un conjunto de \textit{patches} por cada expresión facial dentro de un conjunto de datos de entrenamiento. En la segunda parte del MTSL, extraído el conjunto  de \textit{patches} por cada expresion facial en el conjunto de datos de entrenamiento dos tareas son desarrolladas, el reconocimiento de expresión facial y verificación facial.  Estas dos tareas son integrados para aprender de los \textit{patches} obtenidos con anterioridad. El aprendizaje tiene como objetivo determinar cuales son los específicos \textit{patches} faciales que corresponden a cada expresión facial. Finalmente, un clasificador SVM es entrenado para reconocer las características contenidas en los \textit{patches} para una de las seis expresiones faciales consideradas en este trabajo. 
}

\item{\textbf{“Local gabor binary patterns from three orthogonal planes for automatic facial expression recognition” \cite{almaev2013local}}

Almaev et al.  propone un descriptor dinámico de apariencias \textit{Local Gabor Binary Patterns de Three Ortogonal Planes}(LGBP-TOP) donde, analizando la textura espacial y dinámica, y combinado con filtros de gabor logra un alto nivel de precisión en el reconocimiento de expresiones faciales en tiempo real. Además, el descriptor propuesto es robusto a errores en la alineación rotacional propios de la captura de registros.
}

\item{\textbf{“An integrated approach for efficient analysis of facial expressions” \cite{ghayoumi2014integrated}}

Ghayoumi et al.  propone la integración de tres métodos para el reconocimiento de expresiones faciales. \textit{Locality Sensitive Hashing} (LSH), \textit{Principal Component Analysis} (PCA) y \textit{Linear Discriminant Analysis} (LDA)  son los tres métodos utilizados para el reconocimiento de expresiones faciales en este trabajo. Para reconocer una expresión facial son extraídos los vectores de características de dos regiones más informativas del rostro, la región de los ojos y la región de la boca. El LSH esta compuesto por funciones hash los cuales son usados para mapear los vectores de características extraídos de las imagenes en bloques de colisiones, donde cada bloque esta registrado como la locación de vectores de características pertenecientes a una específica expresión facial a priori. Así, se logra reducir redundancia en el espacio de representación de las imagenes respecto a las imágenes que pertencen al mismo tipo de expresion facial. Luego de este paso son aplicados los métodos de reducción de dimensionalidad PCA y LDA  para aliviar la complejidad computacional y reducir redundancia en los datos. Por último, es entrenado un clasificador SVM con los datos obtenidos despues de aplicado los métodos de reducción de dimensionalidad.
}

\end{enumerate}

\subsection{Técnicas de \textit{Deep Learning}}

Muchos de los avances en visión por computador, procesamiento del lenguaje natural, bioinformática y otros campos de investigación se debe a la utilización de métodos y técnicas de inteligencia artificial, específicamente técnicas de \textit{machine learning} y \textit{deep learning}. Este último, es el causante de una gran impacto en el avance técnologico con relacionamiento directo en el mundo comercial (p.e. carros autónomos, chatboots, creación de nuevos farmacos por computador, etc). 

\begin{enumerate}

\item{\textbf{“Gradient-based learning applied to document recognition” \cite{2lecun1998gradient}}

 Yan Lecunn  et al. introduce uno de los primeros trabajos con redes neuronales convolucionales, un tipo especial de \textit{deep learning}. En este trabajo, muestra la potencialidad de la técnica de aprendizaje basado en gradiente. Esta técnica es utilizado en el entrenamiento de una \textit{multilayer perceptron} via el algoritmo de \textit{backpropagation}. Es diseñado una red neuronal convolucional para tratar con la variabilidad de forma 2D en imágenes para el reconocimiento de dígitos escritos a mano. También es introducido dos sistemas on-line para el reconocimiento de documentos escritos a mano. Para controlar los componentes del sistema tales como la extracción, segmentación, reconocimiento y modelage del lenguaje; es introducido un nuevo paradigma de aprendizaje llamado \textit{Graph Transformer networks}. Además, es descrito la pontencialidad del \textit{Graph transformer networks} aplicado a la lectura de cheques de banco. Donde, las redes neuronales convolucionales para el reconocimiento de caracteres combinado con técnicas de entrenamiento global proporcionaron un alto rendimiento. Esto fue demostrado en el area comercial al ser capaz el modelo de leer cheques personales de forma automatizada, logrando leer millones de cheques por día. 
}
\item{\textbf{“Imagenet classification with deep convolutional neural networks” \cite{8krizhevsky2012imagenet}}
 
Krizhevsky et al. propone una arquitectura de red neuronal convolucional profunda que es entrenado para clasificar 1.2 millones de imágenes en alta resolución en 1000 categorías. La red propuesta en este trabajo llega a estar compuesto por 60 millones de parámetros y 650000 neuronas. Para el entrenamiento fueron usados \textit{GPUs} para acelerar el computo de multiplicaciones matriciales requeridas en las operaciones de convolución. Este trabajo logra el estado del arte en clasificación de imágenes a gran escala en la competición ILSVRC-2012. Esto es demostrado al lograr reducir el error de clasificación de imágenes en la base de datos ImageNet de 26.2\% a 15.3 \%. Siendo así, uno de los trabajos con más impacto en visión por computador. 

Así, es demostrado que una de las principales desventajas de los métodos basados en extracción de características diseñadas a mano es el tiempo de computo lo que les dificulta ser escalables, requisito principal para aplicaciones en el mundo real. además, estos métodos solo cubren una parte específica de casos, esto debido a las suposiciones que se toman para casos específicos que no logran cubrir todo el espectro de posibilidades para poder crear modelos generalizables. Por otro lado, los métodos de \textit{deep learning} demostraron ser capaces de crear modelos generalizables y escalables. Hecho este análisis, fuimos motivados a desarrollar una arquitectura de red neuronal convolucional para el reconocimiento de expresiones faciales.
}
\end{enumerate}
\section{Objetivos}
\subsection{Objetivo General}
Desarrollar una arquitectura de red neuronal convolucional que sea capaz de obtener niveles de precisión confiables (mínimo margen de error) en el reconocimiento de expresiones faciales, permitiendo así contribuir en el desarrollo de futuras aplicaciones del mundo real que sirvan para el beneficio de la sociedad.
\subsection{Objetivos Específicos}
\begin{itemize}
\item Selección de las bases de datos de expresiones faciales y el pre-procesamiento de estos datos (estandarización y normalización).

\item Investigar los filtros de convolución para la correcta selección de los parámetros.

\item Investigar la función de submuestreo para la correcta selección de los parametros. 

\item Investigar las funciones de activación y funciones de normalización para la correcta selección de los parametros.

\item Diseñar la arquitectura propuesta (configuración de parametros, número de capas y funciones de activación y normalización), basandonos en los objetivos previos.

\item Entrenar la arquitectura propuesta.
	
\item Evaluar el modelo creado a partir de la arquitectura propuesta.

\item Analizar e interpretar los resultados
\end{itemize}

\section{Alcances}
En este trabajo de investigación se lograron los siguientes alcances.

\begin{itemize}
\item Se propuso una nueva arquitectura para el reconocimiento de expresiones faciales, basada en las redes neuronal convolucional. El modelo creado fue capaz de obtener altos niveles de precisión que serán de utilidad para el desarrollo de futuras aplicaciones en el mundo real.
\item Se creo una nueva base de datos, la cual fue resultado de la unión de las dos bases de datos antes mencionadas(FER2013 y CK+).
\item Contribuimos con la comunidad académica del país y la región brindandoles información de un tema de investigación actual que servirá como base para el desarrollo de futuras aplicaciones y trabajos relacionados.
\end{itemize}
 

\section{Justificación}
En la actualidad se ha dado más realce a algunas disciplinas de la inteligencia artificial como: \textit{machine learning} y \textit{deep learning}, disciplinas que brindan distintas técnicas que están dando solución a problemas de clasificación de imágenes, comprensión de escena, análisis de sentimientos y otros. Así, es el caso de la visión artificial donde las redes neuronales convolucionales está proporcionando mejores resultados en comparación con algoritmos y técnicas tradicionales.

En este trabajo, presentamos un estudio resumido de la investigación hecha en \textit{deep learning} con aplicación en el reconocimiento de expresiones faciales que servirá tanto para los investigadores como para los lectores. También este trabajo ayudara para el desarrollo de futuros proyectos de clasificación de imágenes en distintos campos (seguridad, medicina y biología, internet y la nube, entretenimiento, máquinas autónomas y otros)\footnote[4]{Aplicaciones de Deep Learning, fuente: \href{https://developer.nvidia.com/deep-learning}{NVIDIA} GPUs - el motor del aprendizaje profundo(deep learning)}.


\section{Metodología}
Dada la naturaleza del trabajo de investigación, se utilizó los métodos de investigación bibliográfica, explorativa y aplicativa. Bibliográfica ya que se recogió y analizó información para obtener conocimientos previos sobre \textit{deep learning} y el detector \textit{haar cascade}. Explorativa porque se seleccionó información relevante procedente de la etapa de investigación bibliográfica, que sirvió para la construcción de la arquitectura de una \textit{red neuronal convolucional} basándonos en trabajos previos relacionados con la línea de investigación. Aplicativa por que se utilizaron los conocimientos adquiridos \cite{19sabino1994hacer}\cite{23silva2001metodologia}.


\section{Limitaciones}
\begin{itemize}
\item Dificil acceso a herramientas tecnológicas de hardware, principalmente GPU's de alta capacidad, necesarias para la fase de entrenamiento de la red neuronal convolucional. 

\item Dificil acceso a información cientifica de fuentes confiables (IEEE, ACM, Springer, etc). Esto debido al elevado costo para acceder a ellos.


\item Carencia de organizaciones peruanas que brinden grandes bases de datos de expresiones faciales para poder utilizarlos en la fase de entrenamiento y crear un modelo para el reconocimiento de expresiones faciales de la región o del país. Así, fuimos llevados a utilizar bases de datos de organizaciones extranjeras que fomentan la investigación en esta área.


% ya que se requiere miles de imágenes (imágenes de acuerdo al proyecto de investigación en el que se trabaje, como: rostros, danzas, señas, sitios arqueológicos y otros) para que se cree un modelo eficiente y robusto, llevando a utilizar base de datos de organizaciones extranjeras que fomentan la investigación en esta área.
\end{itemize}
\newpage
\section{Cronograma de Actividades}
\begin{table}[!htb]
    \centering
    \includegraphics[angle=90,width=50mm]{Imagenes/cronograma.png}
    \caption{Cronograma de actividades}
    \label{tab:tab1}
\end{table}


