\chapter*{CONCLUSIONES}
\begin{itemize}

\item En este trabajo, desarrollamos un clasificador para el reconocimiento de expresiones faciales considerando seis categorías (alegre, neutro, feliz, triste, enojado, sorpresa).

\item Las bases de datos usadas en este trabajo Fer2013 \ref{subsec:fer2013} y CK+ \ref{subsec:ck+}, fueron estandarizados en tamaño y posteriormente normalizados en valor para reducir la varianza en los datos. 

\item Adicionalmente, se creo una tercera base de datos llamado Fer2013-CK+ \ref{subsec:ck+fer2013}, como resultado de la unión de las bases de datos Fer2013 y CK+.  

\item Desarrollamos una arquitectura de red neuronal convolucional para el reconocimiento de expresiones faciales \ref{sec:arq_propuesta}.

\item La arquitectura usada para crear el modelo para el reconocimiento de expresiones faciales fue el resultado de tres arquitecturas propuestas, siendo seleccionada la arquitectura con mejores resultados en los experimentos \ref{sec:experiment}.

\item Cada una de las tres arquitectura propuestas, tuvo una  configuración diferente de los elementos principales que componen una red neuronal convolucional (número de capas de convolución, tamaño de los filtros de convolución, número de capas de pooling, operación de pooling, número de capas de neuronas totalmente conectadas, número de neuronas en la capa totalmente conectada) basada en trabajos de la literatura de \textit{deep learning} mostrados en \ref{sec:experiment}..

\item Se realizó satisfactoriamente los experimentos en las tres bases de datos utilizadas Fer2013, CK+ y Fer2013-CK+ \ref{sec:experiment}.

\item Se evaluó los resultados obtenidos por el modelo creado en las tres bases de datos Fer2013, CK+, Fer2013-CK+. Obteniendo un resultado de 90 \% de precisión en la base de datos CK+, mientras que en las bases de datos Fer2013 y Fer2013-CK+ se obtuvo 57\% y 60\% de precisión respectivamente (más detalles ver Tablas ~\ref{tab:tabla_resultados_ck+}, ~\ref{tab:tabla_resultados_fer}, ~\ref{tab:tabla_resultados_fer_ck+} ).


\item Adicionalmente, una sugerencia fue alterar las imagenes con diferentes rotaciones, iluminación, etc. Sin embargo, esto no representaria las condiciones de las imagenes del mundo real, por lo que nosotros obtamos por recolectar un pequeño conjunto de imagenes de internet con contenido variado (fondo con diferente iluminación, oclusión parcial del rostro, posición rotada en algunos rostros, imagen de animes) que si representan en su mayoria las condiciones de las imágenes del mundo real. Así, se  hicieron pruebas en este conjunto de imágenes de internet. Los resultados obtenidos (pruebas satisfactorias - pruebas fallidas) son mostrados en \ref{sec:testing}.

\end{itemize}