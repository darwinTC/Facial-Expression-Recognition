%\chapter*{Resumen}
%Las expresiones faciales son un medio de comunicación no verbal por el que los seres humanos transmiten sus emociones en el proceso de interacción con su entorno. Esta interaccion  puede darse con otros seres humanos o puede ser una reacción frente a estimulos externos de su entorno como la publicidad o reacción frente a un servicio consumido o por consumir. Existen muchas aplicaciones útiles en el mundo real que son derivadas del reconocimiento automático de expresiones faciales tales como: Estudio de marketing, interacción hombre-máquina, análisis psicologico, seguridad, etc. Muchos abordages tradicionales aplicados al reconocimiento de expresiones faciales tienen dificultad en la extracción y representación de características de la imagen. Esta dificultad se debe a que estos metodos diseñan la extracción de caracteristicas de forma manual y la representación de características no toma en consideración el relacionamiento global dentro de toda la imagen. Tanto la extracción como representación de características de una imagen son parte importante para construir un buen clasificador. Recientemente, las técnicas de deep learning estan logrando resolver las dificultades de los métodos tradicionales en muchas tareas de visión por computador. Especificamente, las redes neuronales convolucionales logran superar el problema de extracción y representación de características de una imagen. En este trabajo, proponemos una arquitectura de red neuronal convolucional para la clasificación de expresiones faciales. Consideramos seis categorías para clasificar las expresiones faciales: Enojo, miedo, alegría, tristeza, sorpresa y neutro. Este trabajo aporta a una mejor comprensión sobre las redes neuronales Convolucionales aplicada al reconocimiento de expresiones faciales e imágenes en general, también ayudara en el desarrollo de futuros proyecto que necesiten del reconocimiento de expresiones faciales.
%\begin{center}
%\noindent\rule{16cm}{0.5pt}
%\end{center}
%\textbf{Palabras Claves:} Expresiones faciales, redes neuronales convolucionales , reconocimiento de Patrones.



\chapter*{Resumen}
Las expresiones faciales son un medio de comunicación no verbal que muestran las emociones de una persona, estas expresiones ayudan a transmitir información en las interacciones inter personales y facilitan el entendimiento del significado del lenguaje hablado. Por lo que se considera que poder clasificar la expresión de un rostro sería una gran fuente de información para una posterior utilización. El objetivo del presente proyecto es modelar el proceso cerebral humano para clasificar imágenes de expresiones faciales por medio de una de las técnicas de \textit{Deep Learning}, logrando así que una máquina sea capaz de aprender de imágenes de expresiones faciales suministradas de ejemplo (datos de entrenamiento) con el objetivo de poder clasificar ejemplos futuros sin ningún tipo de intervención humana en el proceso. En la actualidad, gracias a las Redes Neuronales Convolucionales, se están logrando buenos resultados en la clasificación de imágenes, detección de objetos, comprensión de escena, en comparación con técnicas convencionales, por lo cual en este proyecto se usó la arquitectura de una Red Neuronal Convolucional para clasificar las expresiones faciales, clasificándolas en 6 categorías: enojo, miedo, alegría, tristeza, sorpresa y neutro. Este trabajo aporta a una mejor comprensión en las redes neuronales Convolucionales aplicada al reconocimiento de expresiones faciales e imágenes en general, también ayudara en el desarrollo de futuros proyecto que necesiten del reconocimiento de expresiones faciales, como: estudio de marketing, interacción hombre-máquina, psicología, análisis educativo y otros.
\begin{center}
\noindent\rule{16cm}{0.5pt}
\end{center}
\textbf{Palabras Claves:} Expresión Facial, \textit{Deep Learning}, Convolutional Neural Networks, Visión por Computador, Reconocimiento de Patrones, Detección de Objetos.