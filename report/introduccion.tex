\chapter*{Introducción}
Uno de los razgos psicológicos naturales de los seres humanos es la necesidad de interacción entre ellos. La interacción puede darse a través de la comunicación. Propio de éste, es generada y obtenida información que ellos usan para tomar decisiones durante el proceso de comunicación. Actualmente, gracias al gran avance tecnológico vivimos en una era digital que, posibilitó la trascendencia de la comunicación directa a la comunicación indirecta masiva a través de las redes sociales, videoconferencias, etc. Producto de ello, la cantidad de datos (imágenes, videos, textos, etc) generada por minuto en internet es exorbitante. Surgiendo muchas potenciales aplicaciones (reconocimiento de expresiones faciales, reconocimiento de objetos, detección de objetos, etc) que implican el análisis de imágenes y/o videos por computador. 

Las expresiones faciales juega un rol importante en la comunicación no verbal entre los seres humanos, además de ser el medio por el que se transmite más del 55\% de información en el proceso  de comunicación. Para el reconocimiento de las expresión facial por medio del computador son diseñados modelos basados en la apariencia y modelos geométricos del rostro humano.
Así, en este trabajo abordamos el reconocimiento de expresiones faciales como un problema de clasificación de imagenes de rostros humanos clasificados en 6 categorias (alegre, neutro, feliz, triste, enojado, sorpresa) basados en la apariencia del rostro.

La metodología propuesta  para abordar el problema de reconocimiento de expresiones faciales en imágenes es compuesta por 3 partes. La primera parte, comprende la estandarización y la normalización de las imagenes. En la estandarización, las imagenes son redimensionadas a un tamaño estandar y convertidas a tonos de cinza. La normalización es aplicada para reducir la varianza en las imágenes. Para ambos casos se usa métodos y técnicas de procesamiento de imagenes. En la segunda parte, se extrae las caracteristicas del rostro y se clasifica en las diferentes clases de expresiones faciales (alegre, neutro, feliz, triste, enojado, sorpresa). Para ello, se diseña un conjunto de arquitecturas de redes neuronales convolucionales (CNNs) con diferentes parámetros. Finalmente, se evalua el modelo con mayor rendimiento en 3 base de datos públicos de expresiones faciales (FER2013, CK+) y un adicional creado a partir de la unión de los dos antes mencionados (Fer2013-CK+). 

Adicionalmente, para comprobar el funcionamiento del modelo en imágenes del mundo real, un pequeño conjunto de imágenes de internet es seleccionado por nosotros. Estas imágenes possen contenidos variados desde imágenes de con ruido, rostros con oclusión hasta imágenes con fondo complejo. Primero, es extraido el rostro humano via el detector de rostros \textit{haar cascade} y posteriormente es reconocido su correspondiente expresion facial usando el modelo creado anteriormente.

El desarrollo del presente trabajo puede resumirse en 3 partes.

\begin{itemize}
\item \textbf{Parte I:} Cubre los aspectos generales del problema, describiendo de una manera detallada el problema al cual se quiere dar solución, los trabajos relacionados, los objetivos a alcanzar, la metodología y las limitaciones encontradas en el desarrollo de la investigación.
\item \textbf{Parte II:} Proporciona los fundamentos teóricos necesarios que son vitales para el desarrollo y entendimiento del proyecto.
\item \textbf{Parte III:} Desarrolla y muestra los experimentos realizados con diferentes configuraciones sobre una arquitectura de red neuronal convolucional(CNN). Se describe a detalles el funcionamiento de los métodos elegidos para la deteccién y el reconocimiento de expresiones faciales. Los resultados obtenidos son interpretados en terminos de una métrica de error y precisión, y se proponen trabajos futuros.

 %Parte III
 %Describe el desarrollo del metodo propuesto, asi como los experimentos realizados en las diferentes configuraciones de las arquitecturas de Redes Neurales convolucionales (CNNs) propuestas. Se describe a detalle el funcionamiento del metodo de clasificacion de expresiones faciales. Los resultados obtenidos son interpretados basados en el error y precision del modelo en las diferentes bases de datos de expresiones faciales. Por ultimo las conclusiones, recomendaciones y trabajos futuros son mostrados.
 
 % Para la aplicacion adicional descrita anteriormente, se describe detalladamente el funcionamiento del metodo de deteccion de rostros. 
 
  
\end{itemize}





