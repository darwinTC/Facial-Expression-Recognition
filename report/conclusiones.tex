\chapter*{CONCLUSIONES}
\begin{itemize}
\item El desarrollo de una arquitectura de Red Neuronal Convolucional es muy
compleja, debido a que no se cuenta con fundamentos teóricos para la correcta
selección de parámetros (filtro de convolución, submuestreo, neuronas, etc.). La
arquitectura propuesta muestra resultados con niveles de precisión alto, lo cual da
evidencia que se hizo una correcta selección de las capas y los parámetros que lo
componen.
\item Existen dos formas de recopilación de datos: La primera consiste en crear una
propia base de datos lo cual requiere de tiempo y dinero, la segunda opción y por
la que se optó en este proyecto, consiste en extraer datos de internet de
organizaciones dedicadas al campo de estudio.

\item El uso de 2 capas de convolución con 64 y 32 filtros de tamaños 4x4 y 2x2 pixeles
respectivamente muestra que es una buena selección de parámetros para la
extracción de características de expresiones faciales.

\item El Submuestreo o Pooling cumple funciones importantes relacionadas con el coste
computacional, reduciendo el número de operaciones de computo con la
disminución de las dimensiones de la imagen con el fin de reducir características.
La utilización de 2 capas de Submuestreo de tamaño de agrupación 2x2 pixeles y
con función Max, muestra que es una buena selección de parámetros para la
reducción de características.

\item La función de activación RELU resulta ser la mejor opción para la
implementación de una arquitectura de Red Neuronal por los resultados mostrados
en el estado del arte del Deep Learning.

\item La función de normalización softmax es muy eficiente para la clasificación de
múltiples clases por los resultados mostrados en el estado del arte del Deep
Learning.

\item Se entrenó satisfactoriamente la Red Neuronal Convolucional (basándonos en la
técnica early stopping - Ver anexos), construida a partir de las capas antes
mencionadas, teniendo algunas limitaciones, sea el caso de recursos
computacionales, ocasionando demoras para la fase de entrenamiento, ya que solo
se contó con el uso de CPU mas no de GPU.

\item En la base de datos de datos CK+ se obtuvo un nivel de precisión alto por que las
imágenes muestran rasgos resaltantes de las expresiones faciales los cuales fueron
etiquetados manualmente en esta investigación. En FER2013 se muestra un nivel
de precisión no muy bueno (Tabla 2) por el desbalance de datos en algunas
categorías y para nivelarlos y alcanzar mejores resultados (Tabla 4), una opción
es combinar con otras bases de datos, pero se corre riesgo de que los criterios de
etiquetado de las expresiones faciales difieran.


\end{itemize}