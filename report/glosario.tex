\begin{itemize}
\item \textbf{Convolución:} Operador matemático que transforma 2 funciones en una tercera función.
\item \textbf{Modelo:} Representación abstracta, conceptual, gráfica, física o matemática, de
fenomenos, sistemas o procesos a fin de analizarlos, describirlos, explicarlos, simularlos
y predecirlos.
\item \textbf{Arquitectura}: Técnica y estilo con lo que se diseña, proyecta y construye un modelo.
\item 
\textbf{Gradiente:} Derivada parcial de una funcion respecto a cada variable de está.

\item \textbf{Exabyte:} Unidad de medida de almacenamiento de datos cuyo símbolo es el 'EB' equivalente a $10^18$ bytes.

\item \textbf{Kaggle:} Plataforma online que ofrece a sus usuarios la opción de participar en distintas
competencias cuyo principal tema es el análisis de gran cantidad de datos.


\item \textbf{Deep Learning} (DL), es un conjunto de algoritmos en aprendizaje automático que intenta
modelar abstracciones de alto nivel en datos usando arquitecturas compuestas de
transformaciones no-lineales múltiples.



\item \textbf{Convolutional Neural Network} (CNN) , tipo de red neuronal artificial donde las neuronas
corresponden a campos receptivos de una manera muy similar a las neuronas en la corteza
visual primaria de un cerebro biológico.

\item \textbf{Machine Learning} (ML) , es una rama de la inteligencia artificial cuyo objetivo es
desarrollar técnicas que permitan a las computadoras aprender.


\item \textbf{Red Neuronal Artificial} (RNA), modelos matemáticos, computacionales, artificiales,
ideales de una red neuronal empleados en estadística, psicología cognitiva, e inteligencia
artificial.


\item \textbf{ Graphics Processor Unit} (GPU), es un coprocesador dedicado al procesamiento de
gráficos u operaciones de coma flotante, para aligerar la carga de trabajo del procesador
central.

\item \textbf{ImageNet:} Base de datos de imágenes a gran escala con 1.2M de imágenes.


\end{itemize}