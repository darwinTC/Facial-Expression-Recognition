\chapter*{Introduccion}
Las expresiones faciales son un medio de información no verbal, útil para entender a las personas en situaciones específicas. Para un ser humano el reconocer una expresión facial es una tarea fácil, pero no lo es para un sistema automatizado basado en visión por computador.
Para reconocer una expresión facial, necesitamos detectar el rostro y reconocer que expresión facial que posee.
En el presente trabajo se usa el detector de rostros \textit{Haar Cascade} para la deteccion y Red Neuronal Convolucional(CNN) para el reconocimiento de la expresión facial. Se usó 3 bases de datos (FER2013 y CK+) y una tercera base de datos como resultado de la unión de las 2 bases de datos antes mencionadas.\\

\begin{itemize}
\item \textbf{Parte I:} Cubre los aspectos generales del problema, describiendo de una manera detallada el problema al cual se quiere dar solucion, los trabajos relacionados, los objetivos a alcanzar, la metodologia y las limitaciones encontradas en el desarrollo de la investigacion.
\item \textbf{Parte II:} Proporciona los fundamentos teóricos necesarios que son vitales para el desarrollo y entendimiento del proyecto.
\item \textbf{Parte III:} Desarrolla y muestra los experimentos realizados con diferentes configuraciones sobre una arquitectura de Red Neuronal Convolucional(CNN). Se describe a detalles el funcionamiento de los metodos elegidos para la deteccion y el reconocimiento de expresiones faciales. Los resultados obtenidos son interpretados en terminos de una metrica de error y precision, y se proponen trabajos futuros.
\end{itemize}





